\ccnusetup{
  % 个人信息
  %   注意:\ccnusetup 中不能出现空行
  info = {
    % cover-type = word,
    cover-type = math,
      % 封面的类型
      %   适用学位类型:【本|硕|博】
      %   可选选项:word|math
      %     word:教务处的 word 封面格式
      %     math:数学与统计学学院的封面格式
      %     默认:math
      %
    title = {
      华中师范大学学位论文 \LaTeX{} 模版
    },
      % 中文-标题
      %   适用学位类型:【本|硕|博】
      %   会自动换行,如果换行点不满意,可以用 \\ 手动换行
      %   能概括整个学位论文的中心内容,简明、扼要
      %   论文题目一般不超过25个字,必要时可加副标题(在题目下一行以“——”打头)
      %
    title* = {
      CCNU thesis \LaTeX{} template 
    }, 
      % 英文-标题
      %   适用学位类型:【本|硕|博】
      %
    author = {你的姓名},
      % 中文-作者姓名
      %   适用学位类型:【本|硕|博】
      %
    author* = {Xing Ming},
      % 拼音-作者姓名
      %   适用学位类型:【硕|博】
      %
    supervisor = {教师姓名 \quad 职称},
      % 中文-指导老师姓名+职称
      %   适用学位类型:【本|硕|博】
      %   职称:讲师|副教授|教授|副研究员|研究员
      %
    supervisor*-name = {Jiao shi},
      % 拼音-指导老师姓名
      %   适用学位类型:【硕|博】
      %
    supervisor*-academic-title = {Professor},
      % 英文-指导老师职称
      %   适用学位类型:【硕|博】
      %
    level = {2018级},
      % 年级
      %   适用学位类型:【本】
      %
    student-id = {学号},
      % 学号
      %   适用学位类型:【本】
      %
    department = {数学与统计学学院},
      % 中文-学院名称
      %   适用学位类型:【本|硕|博】
      %
    department* = {School of Mathematics and Statistics},
      % 英文-学院名称
      %   适用学位类型:【硕|博】
      %
    major = {应用统计},
      % 中文-专业
      %   适用学位类型:【本|硕|博】
      %   选项参考:
      %     本:数学与应用数学(试验)|数学与应用数学(师范)|统计学
      %     硕:应用统计
      %   【硕博】「学术型学位学科专业」填写《授予博士、硕士学位和培养研究生的学科、专业目录》中的二级学科或我校自主设置专业;「专业型学位学科专业」填写「专业学位领域」,无领域学科不填
      %
    major* = {Mathematics},
      % 英文-专业
      %   适用学位类型:【硕|博】
      %   选项参考:
      %     硕:Mathematics|Applied Statistics
      %
    research-area = {教育大数据},
      % 中文-研究方向
      %   适用学位类型:【博】
      %   选项参考:
      %     硕:教育大数据
      %
    research-area* = {Education big data},
      % 英文-研究方向
      %   适用学位类型:【硕|博】
      %
    degree-type = {应用统计硕士},
      % 中文-申请学位学生类别
      %   适用学位类型:【硕|博】
      %   选项参考:
      %     硕:教育硕士|应用统计硕士|全日制硕士|同等学力人员|
      %        高校教师在职攻读硕士学位人员|专业学位人员
      %     博:博士
      %
    degree-type* = {M.S.},
      % 英文-申请学位学生类别,缩写
      %   适用学位类型:【硕】
      %   选项参考:
      %     硕:M.S.
      %
    keywords = {
      关键词1,
      关键词2,
      关键词3
    },
      % 中文-论文关键词
      %   适用学位类型:【本|硕|博】
      %   关键词之间用西文逗号 “,” 隔开
      %
    keywords* = {
      keyword1,
      keyword2,
      keyword3
    },
      % 英文-论文关键词
      %   适用学位类型:【本|硕|博】
      %   关键词之间用西文逗号 “,” 隔开
      %
    % year  = {2022},
      % 年份
      %   适用学位类型:【本|硕|博】
      %   如果不手动调整,则默认是「编译时」的年份
      %
    % month = {5},
      % 月份
      %   适用学位类型:【本|硕|博】
      %   如果不手动调整,则默认是「编译时」的月份
  },
  style = {
    font = times,
      % 西文字体
      %   适用学位类型:【本|硕|博】
      %   可选选项:newtx|times|stixtwo|xits|tg|none
      %     目前的字体配置为: (TG = TeX Gyre, 默认值为 times)
      %     *******************************************************************
      %     选项名   : serif,            sans,     mono,         math
      %     *******************************************************************
      %     stixtwo : STIX Two Text,    TG Heros, TG Cursor,    STIX Two Math
      %     xits    : XITS,             TG Heros, TG Cursor,    XITS Math
      %     times   : Times New Roman,  Arial,    Courier New,  newtxmath
      %     newtx   : TG Termes,        TG Heros, TG Cursor,    newtxmath
      %     tg      : TG Termes,        TG Heros, TG Cursor,    TG Termes Math
      %     *******************************************************************
    footnote-style = xits,
      % 脚注编号样式
      %   可选选项:plain|libertinus|libertinus*|libertinus-sans|
      %           pifont|pifont*|pifont-sans|pifont-sans*|
      %           xits|xits-sans|xits-sans*
      %   默认与西文字体保持一致
      %   注意:对于本科生,《附件4:关于毕业论文注释与参考文献著录格式修订的通知》中指出“文科术科的论文注释「使用」脚注”及“理工科的论文注释「不使用」脚注”
    cjk-font = fandol,
      % 中文字体
      %   适用学位类型:【本|硕|博】
      %   可选选项:adobe|fandol|founder|mac|sinotype|sourcehan|windows|none
      %   默认:fandol
      %   注意:
      %     1. 中文字体设置高度依赖于系统。各系统建议方案:
      %          windows:cjk-font = windows
      %          mac:    cjk-font = mac
      %          linux:  cjk-font = fandol(默认值)
      %     2. 除 fandol 和 sourcehan 外,其余字体均为商用字体,请注意版权问题
      %     3. 但 fandol 字体缺字比较严重,而 sourcehan 没有配备楷体和仿宋体
      %
    % chapter-breakstyle = continuous,
    % chapter-breakstyle = newpage,
      % chapter 是否要新起一页开始
      %   适用学位类型:【本】
      %   可选选项:continuous|newpage
      %   默认:continuous
      %     continuous:chapter 不新起一页开始
      %     newpage:chapter 新起一页开始
      %
    caption-labelstyle = hyphen,
      % 图表标题 label 计数样式
      %   适用学位类型:【本|硕|博】
      %     【硕|博】应选择 dot (根据研究生学位论文规范)
      %   可选选项:arabic|hyphen|dot
      %     arabic:样式为图1,图2,表1,表2...,并且跨 chapter 连续编号,即 上一个 chapter 的图编号若为4,下一个 chapter 的第一个图编号为 5
      %     hyphen:样式为图1-1,图2-1,表1-1...
      %       x-y 的 x 为 chapter 值,y 为图表的计数器值,新的 chapter 中 y 会清零重新计数
      %     dot:样式为图1.1,图2.1,表1.1...
      %       x.y 的 x 为 chapter 值,y 为图表的计数器值,新的 chapter 中 y 会清零重新计数
      %     默认:hyphen
      %
    caption-labelseperator = colon,
      % 图表标题 label 和 标题内容 之内的分隔符
      %   适用学位类型:【本|硕|博】
      %   可选选项:colon|space
      %     colon 表示 「:␣」,即一个西文冒号加一个空格
      %     space 表示 「␣␣」,即两个空格
      %     默认:colon
      %
    show-head = true,
    % show-head = false,
      % 是否显示页眉
      %   适用学位类型:【硕|博】
    show-headlogo = true,
      % 是否显示页眉 logo
      %   适用学位类型:【硕|博】
      %   可选选项:true|false
      %     默认:false
      %   logo 内容:华中师范大学校徽 + 硕士/博士学位论文中英文字样
      %
    headline = thick-thin,
      % 页眉的线的类型
      %   适用学位类型:【硕|博】
      %   可选选项:single|double|thin-thick|thick-thin|none
      %     single:一条线
      %     double:两条线,一样粗细
      %     thin-thick:两条线,上细下粗
      %     thick-thin:两条线,上粗下细
      %     none:没有页眉的线
      %     默认:none
      %
    % head-scope = all,
    head-scope = main,
      % 页眉的作用范围
      %   适用学位类型:【硕|博】
      %   可选选项:all|main
      %     all:除了封面外所有
      %     main:在正文开始才有页眉
      %   默认:main
    % keywords-newline = true,
      % 摘要和关键词之间是否空一行
      %   适用学位类型:【本|硕|博】
      %   可选选项:true|false
      %   默认:本:false,硕博:true
      %
    % listoffigures-show = true,
      % 是否显示图目录
      %   适用学位类型:【本|硕|博】
      %   可选选项:true|false
      %   默认:本:false,硕博:true
      %
    % listoftables-show = true,
      % 是否显示表目录
      %   适用学位类型:【本|硕|博】
      %   可选选项:true|false
      %   默认:本:false,硕博:true
      %
    listoffigures-name = {插 \quad 图},
    listoftables-name  = {表 \quad 格},
      % 图表目录的节标题
      %   适用学位类型:【本|硕|博】
      %
    fullwidth-stop = catcode,
      % 标点的自动替换
      %   适用学位类型:【本|硕|博】
      %   可选选项:catcode|mapping|false
      %     catcode:显式的 “。” 会被替换为 “.”(e.g. 不包括用宏定义保存的 “。”)
      %     mapping:所有的 “。” 会被替换为 “.”(使用 LuaLaTeX 编译则无效)
      %     false:不进行替换
      %     默认:catcode
      %   作用:是否把全角实心句点 “.” 作为默认的句号形状,即正文中输入“。” 最终编译效果为“. ”
      %   说明:一般科技类文章需要替换,防止“. ”与“。”混淆
      %
    bib-style = ccnu-bachelor-numerical,
      % bib-style 表示参考文献的格式
      %   适用学位类型:【本|硕|博】
      %   可选选项:
      %     ccnu-bachelor-numerical|ccnu-bachelor-author-year
      %     ccnu-master|ccnu-doctor|gb7714-2015
      %       ccnu-bachelor-numerical:【本】学校标准,顺序编码制
      %       ccnu-bachelor-author-year:【本】学校标准,作者-年制
      %       ccnu-master:【硕】国标,顺序编码制
      %       ccnu-doctor:【博】国标,顺序编码制
      %       gb7714-2015:国标,顺序编码制
      %     默认:ccnu-bachelor-numerical
      %
    bib-resource = {CCNUthesis-main.bib},
      % 参考文献数据源
      %   适用学位类型:【本|硕|博】
      %   注意:需要加 bib 后缀
      %   默认:CCNUthesis-main.bib
      %
    hyperlink = none,
      % 超链接样式
      %   可选选项:color|none
      %   默认:none
      %
    hyperlink-color = default,
      % 超链接颜色
      %   可选选项:default|classic|material|graylevel|prl
      %   默认:default
      %
  }
}