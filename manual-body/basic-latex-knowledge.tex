\section{ \LaTeX{} 基础知识}

\subsection{安装相关}
因为没有用过Linux系统,所以下面仅针对Windows系统和Mac系统展开,Linux系统是类似的。


\subsubsection{ 安装 \TeX{} Live }
安装必读的中文官方文档:\href{https://gitee.com/OsbertWang/install-latex-guide-zh-cn/releases}{install-latex-guide-zh-cn}

这篇文档已经千锤百炼,非常成熟了。其中Windows用户遇到的问题是最多的,Windows用户一定要\emph{先把Windows系统的部分完整认真读完再去跟着安装},要心里有数,自己的电脑是个什么状态。

Mac用户的就很简单了,pkg包下载直接一键安装即可。


\subsubsection{ 安装外置PDF阅读器 }

其实不妨可以先试试编辑器自带的阅读器,如果觉得体验不好再下载也不迟。

前面介绍过SumatraPDF,这是Windows用户推荐使用的,而Mac用户推荐使用的是Skim阅读器。

\subsubsection{ 安装 Visual Studio Code }

虽然你安装完 \TeX{} Live 后会有一个编辑器,但是个人是非常推荐使用Visual Studio Code的,有几个理由:

\begin{enumerate}
  \item Visual Studio Code打开速度快,比 \TeX{}studio 是肉眼可见地快。
  \item 配置非常简单,下载一个插件LaTeX Workshop,把别人弄好的配置文件代码复制粘贴后即可使用。
  \item 详细配置教程详见 \href{https://zhuanlan.zhihu.com/p/38178015?utm_source=qq&utm_medium=social&utm_oi=1122597840500740096}{使用VSCode编写LaTeX}。
\end{enumerate}



\subsection{\LaTeX{}知识补充}

\subsubsection{表格}

\subsubsection{选择题选项排版}
