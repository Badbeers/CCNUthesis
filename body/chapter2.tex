% !TeX root = ../main.tex

\chapter{引用与链接}

\section{脚注}

华中师范大学《附件4:关于修订毕业论文注释与参考文献著录格式的通知》提到

\begin{itemize}
  \item 文科术科的论文注释使用脚注
  \item 理工科的论文注释不使用脚注
\end{itemize}

\subsection{测试}
\subsection{测试}
\subsection{测试}
\subsection{测试}



\section{引用文中小节}\label{sec:ref}

如引用小节~\ref{sec:ref}



\section{引用参考文献}

这是一个参考文献引用的范例:“\parencite{邱泽奇建构与分化}提出……”。还可以引用多个文献:“\parencite{丁文祥rawtype,李晓东rawtype}提出……”。

英文文献 \parencite{feynman2011}

英文文献 \parencite[12]{feynman2011}

英文文献 \parencite[Thm1]{feynman2011}

英文文献 \parencite[12][Thm1]{feynman2011}

英文文献 \cite{feynman2011}

英文文献 \cite[12]{feynman2011}

英文文献 \cite[Thm1]{feynman2011}

英文文献 \cite[12][Thm1]{feynman2011}

\section{链接相关}


模板使用了 hyperref 包处理相关链接,使用 \verb|\href| 可以生成超链接,默认不显示链接颜色。如果需要输出网址,可以使用 \verb|\url| 命令,示例:\url{https://github.com}。


\begin{proof}
  \[
    x^2
  \]
\end{proof}
\begin{proof}
  \[
    x^2  \qedhere
  \]
\end{proof}