% !TeX root = ../main.tex

\chapter{常用命令环境示例}

此章用于展示一些常用的命令环境的效果,用户在具体写论文过程中可以用来参考效果,如果不需要此章出现在正文中,只需要将主文件 \verb|main.tex| 文件中的 \verb|\chapter{常用命令环境示例}

此章用于展示一些常用的命令环境的效果,用户在具体写论文过程中可以用来参考效果,如果不需要此章出现在正文中,只需要将主文件 \verb|main.tex| 文件中的 \verb|\chapter{常用命令环境示例}

此章用于展示一些常用的命令环境的效果,用户在具体写论文过程中可以用来参考效果,如果不需要此章出现在正文中,只需要将主文件 \verb|main.tex| 文件中的 \verb|\chapter{常用命令环境示例}

此章用于展示一些常用的命令环境的效果,用户在具体写论文过程中可以用来参考效果,如果不需要此章出现在正文中,只需要将主文件 \verb|main.tex| 文件中的 \verb|\include{./body/chapter0.tex}| 代码注释掉即可(不建议删除,因为随时可以取消注释查看效果)。


\section{已定义好的一些数学定理环境}

定理环境内的括号,不管是中文还是西文括号,都不会出现倾斜,不需要像旧模版一样需要用手动用 \verb|\textit| 调整
\begin{definition}[测度]
  (参见文献xxx) 这是一段文字 $E = m c^2$  (中文括号)和 (西文括号)
\end{definition}

\begin{theorem}
  这是一段文字 $E = m c^2$
\end{theorem}

\begin{proof}
  这是一段文字 $E = m c^2$
\end{proof}

\begin{proof}[定理xx的证明]
  这是一段文字 $E = m c^2$
\end{proof}

\begin{example}
  这是一段文字 $E = m c^2$
\end{example}

\begin{property}
  这是一段文字 $E = m c^2$
\end{property}

\begin{proposition}
  这是一段文字 $E = m c^2$
\end{proposition}

\begin{corollary}
  这是一段文字 $E = m c^2$
\end{corollary}

\begin{lemma}
  这是一段文字 $E = m c^2$
\end{lemma}

\begin{axiom}
  这是一段文字 $E = m c^2$
\end{axiom}

\begin{antiexample}
  这是一段文字 $E = m c^2$
\end{antiexample}

\begin{conjecture}
  这是一段文字 $E = m c^2$
\end{conjecture}

\begin{question}
  这是一段文字 $E = m c^2$
\end{question}

\begin{claim}
  这是一段文字 $E = m c^2$
\end{claim}

\begin{remark}
  这是一段文字 $E = m c^2$
\end{remark}



\section{浮动体使用}

用 \verb|\label| 引用时,只需要将其放在 \verb|\caption| 的下一行即可。

和定理类环境的引用相同,建议 label 的名称格式为 \verb|figure:xxx| 或 \verb|table:xxx| 其中 \verb|xxx| 可以写中文,尽可能言简意赅地写这个图或表的内容描述,也尽可能写出图表的“独一无二性”,方便自己记忆,也防止在图表一多的时候不知道引用哪一个。

\verb|\figure| 的 \verb|\caption| 是放在 \verb|\includegraphics| 的下方,而 \verb|\table| 的 \verb|\caption| 是放在 \verb|tabular| 或 \verb|tblr| 环境的上方。

\begin{figure}[htbp]
  \centering
  \includegraphics[width = 5cm]{example-image-a}
  \caption{测试}
  \label{figure:test}
\end{figure}

\begin{table}[htbp]
  \centering
  \caption{测试}
  \label{table:test}
  \begin{tabular}{|c|c|}
    11 & 22 \\
    33 & 44 
  \end{tabular}
\end{table}

图 \ref{figure:test} 和表 \ref{table:test} 用来测试两个浮动体和交叉引用



\section{部分数学符号的输入}

本节主要是一些数学符号的输入介绍


\subsection{直体符号}

科技类论文中,建议一些数学符号使用直体(“up”前缀表示直体)
  \begin{itemize}
    \item 直立的 pi :\verb|\uppi| $\to \uppi$
    \item 直立的 e :\verb|\upe| $\to \upe$
    \item 直立的 i :\verb|\upi| $\to \upi$
  \end{itemize}



\subsection{physics 宏包的一些命令}

\verb|CCNUthesis.cls| 已经默认加载了 \verb|physics| 宏包,提供了很多很方便的命令,本小节介绍部分,更多的请阅读宏包文档(命令行输入 \verb|texdoc physics|)

\begin{itemize}
  \item 向量 $\vb{a}, \vb*{a}, \va{a}, \va*{a}, \vu{a}, \vu*{a}$
  \item 积分
    \[
      \int_{0}^{1} x \dd{x},
      \dd[3]{x}, \dd(\cos\theta),
    \]
  \item 微分
    \[
      \dv{x}, \dv{f}{x}, \dv[n]{f}{x}, \dv*{f}{x}
    \]
  \item 偏微分
    \[
      \pdv{x}, \pdv{f}{x}, \pdv[n]{f}{x}
    \]
  \item 绝对值 $\abs{x^3}, \abs{\frac{1}{2}}$
  \item 范数 $\norm{f}, \norm{f_n}_{2}^{2}$
\end{itemize}| 代码注释掉即可(不建议删除,因为随时可以取消注释查看效果)。


\section{已定义好的一些数学定理环境}

定理环境内的括号,不管是中文还是西文括号,都不会出现倾斜,不需要像旧模版一样需要用手动用 \verb|\textit| 调整
\begin{definition}[测度]
  (参见文献xxx) 这是一段文字 $E = m c^2$  (中文括号)和 (西文括号)
\end{definition}

\begin{theorem}
  这是一段文字 $E = m c^2$
\end{theorem}

\begin{proof}
  这是一段文字 $E = m c^2$
\end{proof}

\begin{proof}[定理xx的证明]
  这是一段文字 $E = m c^2$
\end{proof}

\begin{example}
  这是一段文字 $E = m c^2$
\end{example}

\begin{property}
  这是一段文字 $E = m c^2$
\end{property}

\begin{proposition}
  这是一段文字 $E = m c^2$
\end{proposition}

\begin{corollary}
  这是一段文字 $E = m c^2$
\end{corollary}

\begin{lemma}
  这是一段文字 $E = m c^2$
\end{lemma}

\begin{axiom}
  这是一段文字 $E = m c^2$
\end{axiom}

\begin{antiexample}
  这是一段文字 $E = m c^2$
\end{antiexample}

\begin{conjecture}
  这是一段文字 $E = m c^2$
\end{conjecture}

\begin{question}
  这是一段文字 $E = m c^2$
\end{question}

\begin{claim}
  这是一段文字 $E = m c^2$
\end{claim}

\begin{remark}
  这是一段文字 $E = m c^2$
\end{remark}



\section{浮动体使用}

用 \verb|\label| 引用时,只需要将其放在 \verb|\caption| 的下一行即可。

和定理类环境的引用相同,建议 label 的名称格式为 \verb|figure:xxx| 或 \verb|table:xxx| 其中 \verb|xxx| 可以写中文,尽可能言简意赅地写这个图或表的内容描述,也尽可能写出图表的“独一无二性”,方便自己记忆,也防止在图表一多的时候不知道引用哪一个。

\verb|\figure| 的 \verb|\caption| 是放在 \verb|\includegraphics| 的下方,而 \verb|\table| 的 \verb|\caption| 是放在 \verb|tabular| 或 \verb|tblr| 环境的上方。

\begin{figure}[htbp]
  \centering
  \includegraphics[width = 5cm]{example-image-a}
  \caption{测试}
  \label{figure:test}
\end{figure}

\begin{table}[htbp]
  \centering
  \caption{测试}
  \label{table:test}
  \begin{tabular}{|c|c|}
    11 & 22 \\
    33 & 44 
  \end{tabular}
\end{table}

图 \ref{figure:test} 和表 \ref{table:test} 用来测试两个浮动体和交叉引用



\section{部分数学符号的输入}

本节主要是一些数学符号的输入介绍


\subsection{直体符号}

科技类论文中,建议一些数学符号使用直体(“up”前缀表示直体)
  \begin{itemize}
    \item 直立的 pi :\verb|\uppi| $\to \uppi$
    \item 直立的 e :\verb|\upe| $\to \upe$
    \item 直立的 i :\verb|\upi| $\to \upi$
  \end{itemize}



\subsection{physics 宏包的一些命令}

\verb|CCNUthesis.cls| 已经默认加载了 \verb|physics| 宏包,提供了很多很方便的命令,本小节介绍部分,更多的请阅读宏包文档(命令行输入 \verb|texdoc physics|)

\begin{itemize}
  \item 向量 $\vb{a}, \vb*{a}, \va{a}, \va*{a}, \vu{a}, \vu*{a}$
  \item 积分
    \[
      \int_{0}^{1} x \dd{x},
      \dd[3]{x}, \dd(\cos\theta),
    \]
  \item 微分
    \[
      \dv{x}, \dv{f}{x}, \dv[n]{f}{x}, \dv*{f}{x}
    \]
  \item 偏微分
    \[
      \pdv{x}, \pdv{f}{x}, \pdv[n]{f}{x}
    \]
  \item 绝对值 $\abs{x^3}, \abs{\frac{1}{2}}$
  \item 范数 $\norm{f}, \norm{f_n}_{2}^{2}$
\end{itemize}| 代码注释掉即可(不建议删除,因为随时可以取消注释查看效果)。


\section{已定义好的一些数学定理环境}

定理环境内的括号,不管是中文还是西文括号,都不会出现倾斜,不需要像旧模版一样需要用手动用 \verb|\textit| 调整
\begin{definition}[测度]
  (参见文献xxx) 这是一段文字 $E = m c^2$  (中文括号)和 (西文括号)
\end{definition}

\begin{theorem}
  这是一段文字 $E = m c^2$
\end{theorem}

\begin{proof}
  这是一段文字 $E = m c^2$
\end{proof}

\begin{proof}[定理xx的证明]
  这是一段文字 $E = m c^2$
\end{proof}

\begin{example}
  这是一段文字 $E = m c^2$
\end{example}

\begin{property}
  这是一段文字 $E = m c^2$
\end{property}

\begin{proposition}
  这是一段文字 $E = m c^2$
\end{proposition}

\begin{corollary}
  这是一段文字 $E = m c^2$
\end{corollary}

\begin{lemma}
  这是一段文字 $E = m c^2$
\end{lemma}

\begin{axiom}
  这是一段文字 $E = m c^2$
\end{axiom}

\begin{antiexample}
  这是一段文字 $E = m c^2$
\end{antiexample}

\begin{conjecture}
  这是一段文字 $E = m c^2$
\end{conjecture}

\begin{question}
  这是一段文字 $E = m c^2$
\end{question}

\begin{claim}
  这是一段文字 $E = m c^2$
\end{claim}

\begin{remark}
  这是一段文字 $E = m c^2$
\end{remark}



\section{浮动体使用}

用 \verb|\label| 引用时,只需要将其放在 \verb|\caption| 的下一行即可。

和定理类环境的引用相同,建议 label 的名称格式为 \verb|figure:xxx| 或 \verb|table:xxx| 其中 \verb|xxx| 可以写中文,尽可能言简意赅地写这个图或表的内容描述,也尽可能写出图表的“独一无二性”,方便自己记忆,也防止在图表一多的时候不知道引用哪一个。

\verb|\figure| 的 \verb|\caption| 是放在 \verb|\includegraphics| 的下方,而 \verb|\table| 的 \verb|\caption| 是放在 \verb|tabular| 或 \verb|tblr| 环境的上方。

\begin{figure}[htbp]
  \centering
  \includegraphics[width = 5cm]{example-image-a}
  \caption{测试}
  \label{figure:test}
\end{figure}

\begin{table}[htbp]
  \centering
  \caption{测试}
  \label{table:test}
  \begin{tabular}{|c|c|}
    11 & 22 \\
    33 & 44 
  \end{tabular}
\end{table}

图 \ref{figure:test} 和表 \ref{table:test} 用来测试两个浮动体和交叉引用



\section{部分数学符号的输入}

本节主要是一些数学符号的输入介绍


\subsection{直体符号}

科技类论文中,建议一些数学符号使用直体(“up”前缀表示直体)
  \begin{itemize}
    \item 直立的 pi :\verb|\uppi| $\to \uppi$
    \item 直立的 e :\verb|\upe| $\to \upe$
    \item 直立的 i :\verb|\upi| $\to \upi$
  \end{itemize}



\subsection{physics 宏包的一些命令}

\verb|CCNUthesis.cls| 已经默认加载了 \verb|physics| 宏包,提供了很多很方便的命令,本小节介绍部分,更多的请阅读宏包文档(命令行输入 \verb|texdoc physics|)

\begin{itemize}
  \item 向量 $\vb{a}, \vb*{a}, \va{a}, \va*{a}, \vu{a}, \vu*{a}$
  \item 积分
    \[
      \int_{0}^{1} x \dd{x},
      \dd[3]{x}, \dd(\cos\theta),
    \]
  \item 微分
    \[
      \dv{x}, \dv{f}{x}, \dv[n]{f}{x}, \dv*{f}{x}
    \]
  \item 偏微分
    \[
      \pdv{x}, \pdv{f}{x}, \pdv[n]{f}{x}
    \]
  \item 绝对值 $\abs{x^3}, \abs{\frac{1}{2}}$
  \item 范数 $\norm{f}, \norm{f_n}_{2}^{2}$
\end{itemize}| 代码注释掉即可(不建议删除,因为随时可以取消注释查看效果)。


\section{列表环境}


如果要修改 \verb|enumerate| 环境的 label 样式的话:

\begin{enumerate}
  \item 第一项
  \item 第二项
  \item 第三项
  \item 第四项
\end{enumerate}

\begin{enumerate}[1)]
  \item 第一项
  \item 第二项
  \item 第三项
  \item 第四项
\end{enumerate}

\begin{enumerate}[a.]
  \item 第一项
  \item 第二项
  \item 第三项
  \item 第四项
\end{enumerate}

\begin{enumerate}[(A)]
  \item 第一项
  \item 第二项
  \item 第三项
  \item 第四项
\end{enumerate}

test
\begin{enumerate}[(i)]
  \item 第一项
  \item 第二项
  \item 第三项
  \item 第四项
\end{enumerate}

\begin{enumerate}[I]
  \item 第一项
  \item 第二项
  \item 第三项
  \item 第四项
\end{enumerate}

\begin{enumerate}[label = \textbf{断言} \Alph*]
  \item 一般来说使用断言,推荐使用 claim 环境
  \item 但是如果真要有一些断言的层级分化的话
  \item 可以考虑用 enumerate 环境的 label 选项
\end{enumerate}

\begin{enumerate}[\textbf{断言} A]
  \item 1
  \item 2
\end{enumerate}


\section{已定义好的一些数学定理环境}

定理环境内的括号,不管是中文还是西文括号,都不会出现倾斜,不需要像旧模版一样需要用手动用 \verb|\textit| 调整
\begin{definition}[测度]
  (参见文献xxx) 这是一段文字 $E = m c^2$  (中文括号)和 (西文括号)
\end{definition}

\begin{theorem}
  这是一段文字 $E = m c^2$
\end{theorem}


\begin{proof}
  这是一段文字 $E = m c^2$
\end{proof}

\begin{proof}[定理xx的证明]
  这是一段文字 $E = m c^2$
\end{proof}

\begin{example}
  这是一段文字 $E = m c^2$
\end{example}

\begin{property}
  这是一段文字 $E = m c^2$
\end{property}

\begin{proposition}
  这是一段文字 $E = m c^2$
\end{proposition}

\begin{corollary}
  这是一段文字 $E = m c^2$
\end{corollary}

\begin{lemma}
  这是一段文字 $E = m c^2$
\end{lemma}

\begin{axiom}
  这是一段文字 $E = m c^2$
\end{axiom}

\begin{counterexample}
  这是一段文字 $E = m c^2$
\end{counterexample}

\begin{conjecture}
  这是一段文字 $E = m c^2$
\end{conjecture}

\begin{question}
  这是一段文字 $E = m c^2$
\end{question}

\begin{claim}
  这是一段文字 $E = m c^2$
\end{claim}

\begin{remark}
  这是一段文字 $E = m c^2$
\end{remark}

\begin{theorem}[Cauchy]\label{thm:test}
  这是一个定理
  \begin{equation}\label{eq:test1}
    a^2 + b^2 = c^2 \geq 0
  \end{equation}

  \begin{equation}\label{eq:test2}
    a^2 + b^2 = c^2 \geq 0
  \end{equation}
\end{theorem}

我想引用定理~\ref{thm:test} 和公式~\ref{eq:test2}


定理括号测试:

\begin{theorem}
  测试
  \begin{enumerate}
    \item 中文(括号)没输入空格的效果
    \item 中文 (括号) 输入空格的效果
    \item 西文(括号)没输入空格的效果
    \item 西文 (括号) 输入空格的效果
  \end{enumerate}
\end{theorem} 


\begin{proof}
  test
  \[
    a^2 + b^2 = c^2
  \]
\end{proof}

\begin{proof}
  test
  \[
    a^2 + b^2 = c^2  \qedhere
  \]
\end{proof}

\section{浮动体使用}

用 \verb|\label| 引用时,只需要将其放在 \verb|\caption| 的下一行即可。

和定理类环境的引用相同,建议 label 的名称格式为 \verb|figure:xxx| 或 \verb|table:xxx| 其中 \verb|xxx| 可以写中文,尽可能言简意赅地写这个图或表的内容描述,也尽可能写出图表的“独一无二性”,方便自己记忆,也防止在图表一多的时候不知道引用哪一个。

\verb|\figure| 的 \verb|\caption| 是放在 \verb|\includegraphics| 的下方,而 \verb|\table| 的 \verb|\caption| 是放在 \verb|tabular| 或 \verb|tblr| 环境的上方。

\begin{figure}[htbp]
  \centering
  \includegraphics[width = 5cm]{example-image-a}
  \caption{测试}
  \label{figure:test}
\end{figure}

\begin{table}[htbp]
  \centering
  \caption{测试}
  \label{table:test}
  \begin{tabular}{|c|c|}
    11 & 22 \\
    33 & 44 
  \end{tabular}
\end{table}

图 \ref{figure:test} 和表 \ref{table:test} 用来测试两个浮动体和交叉引用



\section{部分数学符号的输入}

本节主要是一些数学符号的输入介绍


\subsection{直体符号}

科技类论文中,建议一些数学符号使用直体(“up”前缀表示直体)
  \begin{itemize}
    \item 直立的 pi :\verb|\uppi| $\to \uppi$
    \item 直立的 e :\verb|\upe| $\to \upe$
    \item 直立的 i :\verb|\upi| $\to \upi$
  \end{itemize}


\section{参考文献引用}

\subsection{数学类}

行间\parencite[thm 3.1]{zurek2014quantum}

行间\parencite{zurek2014quantum}



\subsection{文科类}

上标\cite[test]{zurek2014quantum}

上标\cite{zurek2014quantum}



\section{《附件4:关于修订毕业论文注释与参考文献著录格式的通知》中的参考文献效果}

  text\parencite{李晓东rawtype}

  text\parencite{Ahnrawtype}

  text\parencite{Ahnrawtype}

  text\parencite{丁文祥rawtype}

  text\parencite{邱泽奇会议论文集rawtype}

  text\parencite{雷光春rawtype}

  text\parencite{zhangrawtype}

  text\parencite{邱泽奇会议论文rawtype}

  text\parencite{马克思rawtype}

  text\parencite{昂温rawtype}

  text\parencite{Fothrawtype}

  text\parencite{杨国枢rawtype}

  text\parencite{Morisonrawtype}

  text\parencite{张志祥rawtype}

  text\parencite{徐秀英rawtype}

  text\parencite{Aldemitarawtype}

  text\parencite{张凯军rawtype}

  text\parencite{Kosekrawtype}

  text\parencite{文献编写rawtype}

  text\parencite{国防白皮rawtype}

  text\parencite{federalrawtype}

  text\parencite{healthrawtype}

  text\parencite{江向东rawtype}

  text\parencite{萧钮rawtype}

  text\parencite{Dublinrawtype}

文字测试