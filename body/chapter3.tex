\chapter{问题研究}


\section{已定义好的一些定理环境}


\begin{definition}
  我是定义
\end{definition}

\begin{theorem}
  我是定理
\end{theorem}

\begin{proof}
  我是证明
\end{proof}

\begin{corollary}
  我是推论
\end{corollary}

\begin{axiom}
  我是公理
\end{axiom}


\begin{example}
  我是例
\end{example}

\begin{lemma}
  我是引理
\end{lemma}

析出文献,即从著作或公开发表的书籍文章中分析出来所获得的文献资料。

连续出版物指印刷或非印刷形式的出版物,具有统一的题名,定期或不定期以连续分册形式出版,有卷期或年月标识,并且计划无限期地连续出版。连续出版物包括:期刊、报纸、年度出版物(年鉴、指南等)以及成系列的报告、学会会刊、会议录和专著丛书等。
%%下面的文献引用用来测试前五个文献类别
\parencite{李晓东,Ahn,丁文祥,张启发,雷光春,邱泽奇,zhang,唐绪军,昂温,Foth,杨国枢,Morison,张志祥,Aldemita,张凯军,Kosek,文献编写,国防白皮,federal,health,江向东,萧钮,PACS-L}
