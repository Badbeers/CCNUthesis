% !TeX root = ../CCNUthesis-doc.tex

\subsection{为什么要用 \LaTeX{}?}

以下主要部分来自 \href{https://ctan.math.illinois.edu/info/lshort/chinese/lshort-zh-cn.pdf}{lshort-zh-cn} 的 1.1.3 \LaTeX{} 的优缺点。

经常有人喜欢对比 \LaTeX{} 和以 Microsoft Office Word 为代表的“所见即所得”(What You See Is What You Get)字处理工具。这种对比是没有意义的,因为 \TeX{} 是一个排版引擎,\LaTeX{} 是其封装,而 Word 是字处理工具。二者的设计目标不一致,也各自有自己的适用范围。

不过,这里仍旧总结 \LaTeX{} 的一些优点:
\begin{itemize}
  \item 具有专业的排版输出能力,产生的文档看上去就像“印刷品”一样。
  \item 具有方便而强大的数学公式排版能力,无出其右者。
  \item 绝大多数时候,用户只需专注于一些组织文档结构的基础命令,无需(或很少)操心文档的版面设计。\emph{这本质上体现了 \LaTeX{} 最重要的思想之一:内容与样式分离(更多关于内容与样式分离的内容可以看 \href{https://liam.page/2019/03/18/separation-of-content-and-presentation/}{到底什么是「内容与样式分离」}一文)},也是 \cls{CCNUthesis} 模版的最基本但也最重要的处理思想:格式由开发者处理,用户只需要关注内容。
  \item 很容易生成复杂的专业排版元素,如脚注、交叉引用、参考文献、目录等。
  \item 强大的可扩展性。世界各地的人开发了数以千计的 \LaTeX{} 宏包用于补充和扩展 \LaTeX{} 的功能。
  \item 能够促使用户写出结构良好的文档——而这也是 \LaTeX{} 存在的初衷。
  \item \LaTeX{} 和 \TeX{} 及相关软件是跨平台、免费、开源的。无论用户使用的是 Windows,macOS(OS X),GNU/Linux 还是 FreeBSD 等操作系统,都能轻松获得和使用这一强大的排版工具,并且获得稳定的输出。
\end{itemize}