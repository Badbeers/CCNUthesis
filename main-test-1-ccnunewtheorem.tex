\documentclass{CCNUthesis}
\ccnunewtheorem{测试}{test}
\ccnunewtheorem*{测试试}{testt}
\ccnunewtheorem[sibling = theorem]{测试试试}{testtt}
\ccnunewtheorem[within = chapter]{测试试试试}{testttt}

\begin{document}

\chapter{1}

\begin{theorem}
  测试
\end{theorem}

\begin{test}
  环境自己用自己的计数器,并且跨 chapter 连续编号
\end{test}

\begin{test}
  环境自己用自己的计数器,并且跨 chapter 连续编号
\end{test}

\begin{testt}
  不编号
\end{testt}

\begin{testt}
  不编号
\end{testt}

\begin{testtt}
  和 theorem 环境共用一个计数器
\end{testtt}

\begin{testtt}
  和 theorem 环境共用一个计数器
\end{testtt}

\begin{testttt}
  和章节有关,并且子计数器随新章节清零重新计数
\end{testttt}

\begin{testttt}
  和章节有关,并且子计数器随新章节清零重新计数
\end{testttt}


\chapter{2}

\begin{theorem}
  测试
\end{theorem}

\begin{test}
  环境自己用自己的计数器,并且跨 chapter 连续编号
\end{test}

\begin{test}
  环境自己用自己的计数器,并且跨 chapter 连续编号
\end{test}

\begin{testt}
  不编号
\end{testt}

\begin{testt}
  不编号
\end{testt}

\begin{testtt}
  和 theorem 环境共用一个计数器
\end{testtt}

\begin{testtt}
  和 theorem 环境共用一个计数器
\end{testtt}

\begin{testttt}
  和章节有关,并且子计数器随新章节清零重新计数
\end{testttt}

\begin{testttt}
  和章节有关,并且子计数器随新章节清零重新计数
\end{testttt}

\end{document}