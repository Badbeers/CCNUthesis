\chapter{研究设计}


\section{研究问题}

本文在总结归纳前人研究的基础上,提出了以下两点研究问题:
\begin{enumerate}
  \item 在复习课上,不同教师的课堂提问类型、频率是否有区别?如果有,区别是什么?
  \item 基于不同的数学课型,教师应当首要遵循哪些课堂提问原则?
\end{enumerate}



\section{研究方法}

根据本次研究的内容,限于目前疫情形势,本次研究采用以下研究方法:


\subsection{文献研究法}

通过各种网站如知网、读秀以及图书馆,收集并分析关于课堂提问的相关文献,了解关于课堂提问的历史以及现状,整理现有阶段的研究成果,为本文的研究提供有效依据。


\subsection{录像分析法}

基于复习课这一课型,本研究选取两位老师的教学视频,反复观看录像,对提问类型、提问方式等进行对比分析,从而探寻课堂提问的策略以提高数学课堂教学质量。



\section{研究过程}

首先,笔者进行大量的文献阅读,确定数学课堂提问的类型、原则;然后,选取两位教师的录像视频,反复观看录像,将教学过程以文本形式记录下来;接着,对课堂提问类型、方式进行统计比较,由此得出一些启示;最后,分析数学复习课堂提问设计的首要原则,并举出相应案例。


\subsection{研究对象}

笔者从一师一优课网站上选取河北省教师A与青海省教师B的课堂实录,授课内容为《函数的应用(复习课)》。这两节课都是部级优课,相信能从中有所启发。


\subsection{研究数据分析}

根据提问的作用以及认知水平的不同,笔者在布鲁姆所提出的记忆、理解、应用、分析、综合、评价这六个层次的基础上,将教师课堂提问区分为四类:
\begin{enumerate*}
  \item 回忆型问题
  \item 理解型问题
  \item 分析型问题
  \item 评价型问题。
\end{enumerate*}
部分研究学者分类中包含课堂管理型提问等等,但本文研究的是高中课堂,教师较少用提问管理课堂记录,因此不作分析。对所记录的课堂提问进行分类整理。