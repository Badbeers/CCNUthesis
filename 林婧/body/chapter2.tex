\chapter{文献综述}


\section{数学课堂提问的功能}

国内外许多学者都对提问的功能与作用提出了自己的理论。

1967年,心理学家帕特对190多位小学教师进行访谈,要求老师们说明各自的提问理由,总结了提问的5种功能:检查理解、帮助教学、诊断困难、促进记忆和刺激思考。这是首次对提问的功能进行总结。1999年,施良方等人在《教学理论:课堂教学的原理、策略与研究》中提出提问的4种功能:诱发学生参与教学、提供线索、提供练习与反馈机会、有助于学生学习结果的迁移\cite{施良方1999教学理论}。2002年,加里在他的《有效教学方法》中总结了提问的6种功能:吸引兴趣和吸引注意力、发现问题及检查、回忆具体知识或信息、课堂管理、鼓励更高层次的思维活动、组织或指导学习\cite{鲍里奇2002有效教学方法}。同年,2013年,朱杰珍在《刍议数学课堂提问的功能与技巧》中归纳数学课堂提问的5种功能:温故导学、激趣探疑、聚精促思、辨析明理、反馈调控\cite{朱杰珍2013刍议数学课堂提问的功能与技巧}。

可以看出,学者们都认为提问对教师和学生双方都是有利的,教师可以利用提问的教学手段帮助教学,学生也能因为教师提问而进行更高层次的思维活动。



\section{数学课堂提问的类型}

1969年,巴恩斯首次提出提问的分类:事实型问题、推理型问题、无需进行推理的开放式问题、社交型问题。而后,巴恩斯又提出推理型问题可分为封闭推理与开放推理两类,其中封闭推理型问题有明确答案,而开放推理型问题没有明确答案。另一种传播甚广的分类方式是由布鲁姆提出的,他根据问题的认知层次将问题划分为6种类型:记忆、理解、应用、分析、综合、评价。后续学者们的研究大多是以此为基础的。对于小学数学课堂,我国学者王小清分析归纳了四种课堂提问类型:猜测型提问、比较型提问、启发型提问、突破型提问\cite{王小清2020小学数学课堂提问类型探究}。2017年,喻洋也对中学数学课堂提问类型进行分类:引入型问题、疏导型问题、探究型问题、总结型问题、感想型问题\cite{喻洋2017中学数学课堂提问的类型与技巧}。

上述文献对课堂提问的分类进行研究,通常是根据提问的功能或者提问的方式两种视角进行研究,在加深对提问的认识上起到了关键作用,熟悉不同提问类型的特点有助于教师根据不同目标选取设计合适的提问类型。



\section{数学课堂提问的原则}

梁平在《初中数学课堂提问有效性及其策略的研究》一文中,针对目前初中数学课堂的提问现状,提出了课堂提问必须遵守的九大原则:目的性原则——精心设计;科学性原则——难易适度;趣味性原则——新颖别致;灵活性原则——因势利导;启发性原则——循循善诱;鼓励性原则——正确评价;广泛性原则——面向全体;针对性原则——因材施教;适量性原则——适量适度\cite{梁平2011初中数学课堂提问有效性及其策略的研究}。

黄小安在《高中数学课堂提问有效性研究》一文中,结合阎承利学者提出的课堂提问的一般性原则,进一步提出具有高中数学学科特征的课堂提问原则:目的性原则;科学性原则;趣味性原则;启发性原则;灵活性和广泛性原则;鼓励性和针对性原则\cite{黄小安2006高中数学课堂提问有效性研究}。



\section{数学课堂提问的有效性}

随着对课堂提问研究的不断深入,学者们开始提出问题:如何检验课堂提问这一教学手段是否真正发挥其功能与作用?有效课堂提问这一概念随之而来。但是目前国内还未有统一的对有效课堂提问的认识。

2003年,赵敏霞通过课堂观察及文献研究发现目前课堂提问现状不如人意,由此她认为有效课堂提问应具有3个特点:是师生交流的重要形式、是实现教学目标整合的重要手段、是服务于学生的学习过程的\cite{赵敏霞2003对教师有效课堂教学提问的思考}。2006年,王雪梅提出有效提问指教师清楚明了地、有目的、有组织地提出简短的、发人深省的问题,要能引起学生的回应和回答\cite{王雪梅2006课堂提问的有效性及其策略研究}。2010年,卢正芝在《教师有效课堂提问:价值取向与标准建构》中提出有效课堂提问的内涵。广义的有效课堂提问是指教师在精心预设问题的基础上,通过创设良好的问题情境,在教学中生成适切的问题,引导学生主动思考,进行质疑和对话,全面实现预期教学目标,并对提问及时反思与实践的过程\cite{卢正芝2010价值取向与标准建构}。2011年,温建红在《数学课堂有效提问的内涵及特征》中提出了有效提问的几大特征:目的性、启发性、多样性、方法性、示范性、情感性\cite{温建红2011数学课堂有效提问的内涵及特征}。

综合以上种种研究,大家探讨课堂提问时大多针对普通教育而言,以数学教学为研究对象的研究比较少,而对复习课这一数学课型的课堂提问研究更是少之又少。因此,我希望对复习课这种数学课型进行探究,进一步分析在特定的数学课型当中课堂提问应遵循何种原则,希望对目前数学教学提供一定帮助。