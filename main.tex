\documentclass{CCNUthesis}

\ccnusetup{
  % 个人信息
  info = {
    % 主标题,会自动换行
      % 如果换行点不满意,可以用 \\ 手动换行
    title = {
      中文主标题
    },
    % 副标题,如果有的话取消下面三行代码的注释并且填写内容
    % subtitle = {
      % 这是一个副标题
    % },
    % 论文的英文标题
    title* = {
      Thesis Title
    }, 
    % 学院
    department = {数学与统计学学院},
    % 专业
    major = {数学与应用数学(试验)},
    % 年级
    level = {2018},
    % 姓名
    author = {你的姓名},
    % 学号
    student-id = {学号},
    % 指导教师
    supervisor = {xxx \quad 教授},
    % 论文中文关键词,用英文“,”隔开
    keywords = {
      关键词1,
      关键词2,
      关键词3
    },
    % 论文英文关键词,用英文“,”隔开
    keywords* = {
      keyword1,
      keyword2,
      keyword3
    },
    % 如有需要可以手动调整封面底的年和月,否则默认为「编译时」的年和月
    % year    = {2022},
    % month = {4}
  },
  style = {
    cjk-font = fandol,
      % 中文字体
      % 允许选项:
      %   cjk-font = adobe|fandol|founder|mac|sinotype|sourcehan|windows|none
      % 注意:
      %   1. 中文字体设置高度依赖于系统。各系统建议方案:
      %        windows:cjk-font = windows
      %        mac:    cjk-font = mac
      %        linux:  cjk-font = fandol(默认值)
      %   2. 除 fandol 和 sourcehan 外,其余字体均为商用字体,请注意版权问题
      %   3. 但 fandol 字体缺字比较严重,而 sourcehan 没有配备楷体和仿宋体
    bib-resource = {CCNUthesis-main.bib}
      % 参考文献数据源,需要加bib后缀
  }
}

%%%%% 需要的宏包可以在此处自行调用 %%%%%



% 需要的命令环境可以自行定义
\newcommand{\upe}{\symrm{e}}        % 直立的e,用于表示常量,如自然常数      
\newcommand{\upi}{\symrm{i}}        % 直立的i,用于表示常量,如虚数单位
\newcommand{\hilbertH}{\symcal{H}}



\begin{document}

\ccnusetup{
  style = {fullwidth-stop = catcode}
    % 是否把全角实心句点 “.” 作为默认的句号形状,即正文中输入“。” 最终编译效果为“. ”
    % 一般科技类文章需要替换,防止“. ”与“。”混淆
    % 【放在此处是为了防止干扰版权页的句号设置】
    % 允许选项:
    %   fullwidth-stop = catcode|mapping|false
    % 说明:
    %   catcode   显式的 “。” 会被替换为 “.”(e.g. 不包括用宏定义保存的 “。”)
    %   mapping   所有的 “。” 会被替换为 “.”(使用 LuaLaTeX 编译则无效)
    %   false     不进行替换
}



% \frontmatter开启论文前置部分
% 前置部分包含目录、中英文摘要以及符号表等
\frontmatter


% 目录
\tableofcontents


% 中文摘要
\begin{abstract}
  中文摘要
\end{abstract}

% 英文摘要
\begin{abstract*}
  English abstract
\end{abstract*}


% 符号表
% 语法与 LaTeX 表格一致:列用 & 区分,行用 \\ 区分
% 如需修改格式,可以使用可选参数:
%   \begin{notation}[ll]
%     $x$ & 坐标 \\
%     $p$ & 动量
%   \end{notation}
% 可选参数与 LaTeX 标准表格的列格式说明语法一致
% 这里的 “ll” 表示两列均为自动宽度,并且左对齐
% 注:如果不需要符号表的话就把"\begin{notation}...\end{notation}"注释掉
\begin{notation}[ll]
  $x$                  & 坐标        \\
  $p$                  & 动量        \\
  $\psi(x)$            & 波函数      \\
\end{notation}
乘积为 $\frac{4}{9}$ 时,乘积为 $4 / 9$ 时



% \mainmatter进入论文主体部分
% 主体部分是论文的核心
\mainmatter

% 主体采用多文件编译的方式
% 即把每一章放进一个单独的 tex 文件里,并在这里用 \include 导入
% 例如 \chapter{引言}

\section{研究背景}
\begin{figure}[tbh]
  \centering
  \includegraphics[width = 5cm]{example-image-a}
  \caption{测试}
\end{figure}
\begin{table}[tbh]
  \caption{测试}
  \centering
  \begin{tblr}{|c|c|}
    11 & 22 \\
    33 & 44 
  \end{tblr}
\end{table}



\subsection{前人工作}


测试 \parencite{邱泽奇建构与分化}
\section{研究背景}

% 表示插入 main 所在目录中的 body 目录下的 chapter1.tex 文件


\chapter{常用命令环境示例}

此章用于展示一些常用的命令环境的效果,用户在具体写论文过程中可以用来参考效果,如果不需要此章出现在正文中,只需要将主文件 \verb|main.tex| 文件中的 \verb|\chapter{常用命令环境示例}

此章用于展示一些常用的命令环境的效果,用户在具体写论文过程中可以用来参考效果,如果不需要此章出现在正文中,只需要将主文件 \verb|main.tex| 文件中的 \verb|\chapter{常用命令环境示例}

此章用于展示一些常用的命令环境的效果,用户在具体写论文过程中可以用来参考效果,如果不需要此章出现在正文中,只需要将主文件 \verb|main.tex| 文件中的 \verb|\include{./body/chapter0.tex}| 代码注释掉即可(不建议删除,因为随时可以取消注释查看效果)。


\section{已定义好的一些数学定理环境}

定理环境内的括号,不管是中文还是西文括号,都不会出现倾斜,不需要像旧模版一样需要用手动用 \verb|\textit| 调整
\begin{definition}[测度]
  (参见文献xxx) 这是一段文字 $E = m c^2$  (中文括号)和 (西文括号)
\end{definition}

\begin{theorem}
  这是一段文字 $E = m c^2$
\end{theorem}

\begin{proof}
  这是一段文字 $E = m c^2$
\end{proof}

\begin{proof}[定理xx的证明]
  这是一段文字 $E = m c^2$
\end{proof}

\begin{example}
  这是一段文字 $E = m c^2$
\end{example}

\begin{property}
  这是一段文字 $E = m c^2$
\end{property}

\begin{proposition}
  这是一段文字 $E = m c^2$
\end{proposition}

\begin{corollary}
  这是一段文字 $E = m c^2$
\end{corollary}

\begin{lemma}
  这是一段文字 $E = m c^2$
\end{lemma}

\begin{axiom}
  这是一段文字 $E = m c^2$
\end{axiom}

\begin{antiexample}
  这是一段文字 $E = m c^2$
\end{antiexample}

\begin{conjecture}
  这是一段文字 $E = m c^2$
\end{conjecture}

\begin{question}
  这是一段文字 $E = m c^2$
\end{question}

\begin{claim}
  这是一段文字 $E = m c^2$
\end{claim}

\begin{remark}
  这是一段文字 $E = m c^2$
\end{remark}



\section{浮动体使用}

用 \verb|\label| 引用时,只需要将其放在 \verb|\caption| 的下一行即可。

和定理类环境的引用相同,建议 label 的名称格式为 \verb|figure:xxx| 或 \verb|table:xxx| 其中 \verb|xxx| 可以写中文,尽可能言简意赅地写这个图或表的内容描述,也尽可能写出图表的“独一无二性”,方便自己记忆,也防止在图表一多的时候不知道引用哪一个。

\verb|\figure| 的 \verb|\caption| 是放在 \verb|\includegraphics| 的下方,而 \verb|\table| 的 \verb|\caption| 是放在 \verb|tabular| 或 \verb|tblr| 环境的上方。

\begin{figure}[htbp]
  \centering
  \includegraphics[width = 5cm]{example-image-a}
  \caption{测试}
  \label{figure:test}
\end{figure}

\begin{table}[htbp]
  \centering
  \caption{测试}
  \label{table:test}
  \begin{tabular}{|c|c|}
    11 & 22 \\
    33 & 44 
  \end{tabular}
\end{table}

图 \ref{figure:test} 和表 \ref{table:test} 用来测试两个浮动体和交叉引用



\section{部分数学符号的输入}

本节主要是一些数学符号的输入介绍


\subsection{直体符号}

科技类论文中,建议一些数学符号使用直体(“up”前缀表示直体)
  \begin{itemize}
    \item 直立的 pi :\verb|\uppi| $\to \uppi$
    \item 直立的 e :\verb|\upe| $\to \upe$
    \item 直立的 i :\verb|\upi| $\to \upi$
  \end{itemize}



\subsection{physics 宏包的一些命令}

\verb|CCNUthesis.cls| 已经默认加载了 \verb|physics| 宏包,提供了很多很方便的命令,本小节介绍部分,更多的请阅读宏包文档(命令行输入 \verb|texdoc physics|)

\begin{itemize}
  \item 向量 $\vb{a}, \vb*{a}, \va{a}, \va*{a}, \vu{a}, \vu*{a}$
  \item 积分
    \[
      \int_{0}^{1} x \dd{x},
      \dd[3]{x}, \dd(\cos\theta),
    \]
  \item 微分
    \[
      \dv{x}, \dv{f}{x}, \dv[n]{f}{x}, \dv*{f}{x}
    \]
  \item 偏微分
    \[
      \pdv{x}, \pdv{f}{x}, \pdv[n]{f}{x}
    \]
  \item 绝对值 $\abs{x^3}, \abs{\frac{1}{2}}$
  \item 范数 $\norm{f}, \norm{f_n}_{2}^{2}$
\end{itemize}| 代码注释掉即可(不建议删除,因为随时可以取消注释查看效果)。


\section{已定义好的一些数学定理环境}

定理环境内的括号,不管是中文还是西文括号,都不会出现倾斜,不需要像旧模版一样需要用手动用 \verb|\textit| 调整
\begin{definition}[测度]
  (参见文献xxx) 这是一段文字 $E = m c^2$  (中文括号)和 (西文括号)
\end{definition}

\begin{theorem}
  这是一段文字 $E = m c^2$
\end{theorem}

\begin{proof}
  这是一段文字 $E = m c^2$
\end{proof}

\begin{proof}[定理xx的证明]
  这是一段文字 $E = m c^2$
\end{proof}

\begin{example}
  这是一段文字 $E = m c^2$
\end{example}

\begin{property}
  这是一段文字 $E = m c^2$
\end{property}

\begin{proposition}
  这是一段文字 $E = m c^2$
\end{proposition}

\begin{corollary}
  这是一段文字 $E = m c^2$
\end{corollary}

\begin{lemma}
  这是一段文字 $E = m c^2$
\end{lemma}

\begin{axiom}
  这是一段文字 $E = m c^2$
\end{axiom}

\begin{antiexample}
  这是一段文字 $E = m c^2$
\end{antiexample}

\begin{conjecture}
  这是一段文字 $E = m c^2$
\end{conjecture}

\begin{question}
  这是一段文字 $E = m c^2$
\end{question}

\begin{claim}
  这是一段文字 $E = m c^2$
\end{claim}

\begin{remark}
  这是一段文字 $E = m c^2$
\end{remark}



\section{浮动体使用}

用 \verb|\label| 引用时,只需要将其放在 \verb|\caption| 的下一行即可。

和定理类环境的引用相同,建议 label 的名称格式为 \verb|figure:xxx| 或 \verb|table:xxx| 其中 \verb|xxx| 可以写中文,尽可能言简意赅地写这个图或表的内容描述,也尽可能写出图表的“独一无二性”,方便自己记忆,也防止在图表一多的时候不知道引用哪一个。

\verb|\figure| 的 \verb|\caption| 是放在 \verb|\includegraphics| 的下方,而 \verb|\table| 的 \verb|\caption| 是放在 \verb|tabular| 或 \verb|tblr| 环境的上方。

\begin{figure}[htbp]
  \centering
  \includegraphics[width = 5cm]{example-image-a}
  \caption{测试}
  \label{figure:test}
\end{figure}

\begin{table}[htbp]
  \centering
  \caption{测试}
  \label{table:test}
  \begin{tabular}{|c|c|}
    11 & 22 \\
    33 & 44 
  \end{tabular}
\end{table}

图 \ref{figure:test} 和表 \ref{table:test} 用来测试两个浮动体和交叉引用



\section{部分数学符号的输入}

本节主要是一些数学符号的输入介绍


\subsection{直体符号}

科技类论文中,建议一些数学符号使用直体(“up”前缀表示直体)
  \begin{itemize}
    \item 直立的 pi :\verb|\uppi| $\to \uppi$
    \item 直立的 e :\verb|\upe| $\to \upe$
    \item 直立的 i :\verb|\upi| $\to \upi$
  \end{itemize}



\subsection{physics 宏包的一些命令}

\verb|CCNUthesis.cls| 已经默认加载了 \verb|physics| 宏包,提供了很多很方便的命令,本小节介绍部分,更多的请阅读宏包文档(命令行输入 \verb|texdoc physics|)

\begin{itemize}
  \item 向量 $\vb{a}, \vb*{a}, \va{a}, \va*{a}, \vu{a}, \vu*{a}$
  \item 积分
    \[
      \int_{0}^{1} x \dd{x},
      \dd[3]{x}, \dd(\cos\theta),
    \]
  \item 微分
    \[
      \dv{x}, \dv{f}{x}, \dv[n]{f}{x}, \dv*{f}{x}
    \]
  \item 偏微分
    \[
      \pdv{x}, \pdv{f}{x}, \pdv[n]{f}{x}
    \]
  \item 绝对值 $\abs{x^3}, \abs{\frac{1}{2}}$
  \item 范数 $\norm{f}, \norm{f_n}_{2}^{2}$
\end{itemize}| 代码注释掉即可(不建议删除,因为随时可以取消注释查看效果)。


\section{已定义好的一些数学定理环境}

定理环境内的括号,不管是中文还是西文括号,都不会出现倾斜,不需要像旧模版一样需要用手动用 \verb|\textit| 调整
\begin{definition}[测度]
  (参见文献xxx) 这是一段文字 $E = m c^2$  (中文括号)和 (西文括号)
\end{definition}

\begin{theorem}
  这是一段文字 $E = m c^2$
\end{theorem}

\begin{proof}
  这是一段文字 $E = m c^2$
\end{proof}

\begin{proof}[定理xx的证明]
  这是一段文字 $E = m c^2$
\end{proof}

\begin{example}
  这是一段文字 $E = m c^2$
\end{example}

\begin{property}
  这是一段文字 $E = m c^2$
\end{property}

\begin{proposition}
  这是一段文字 $E = m c^2$
\end{proposition}

\begin{corollary}
  这是一段文字 $E = m c^2$
\end{corollary}

\begin{lemma}
  这是一段文字 $E = m c^2$
\end{lemma}

\begin{axiom}
  这是一段文字 $E = m c^2$
\end{axiom}

\begin{antiexample}
  这是一段文字 $E = m c^2$
\end{antiexample}

\begin{conjecture}
  这是一段文字 $E = m c^2$
\end{conjecture}

\begin{question}
  这是一段文字 $E = m c^2$
\end{question}

\begin{claim}
  这是一段文字 $E = m c^2$
\end{claim}

\begin{remark}
  这是一段文字 $E = m c^2$
\end{remark}



\section{浮动体使用}

用 \verb|\label| 引用时,只需要将其放在 \verb|\caption| 的下一行即可。

和定理类环境的引用相同,建议 label 的名称格式为 \verb|figure:xxx| 或 \verb|table:xxx| 其中 \verb|xxx| 可以写中文,尽可能言简意赅地写这个图或表的内容描述,也尽可能写出图表的“独一无二性”,方便自己记忆,也防止在图表一多的时候不知道引用哪一个。

\verb|\figure| 的 \verb|\caption| 是放在 \verb|\includegraphics| 的下方,而 \verb|\table| 的 \verb|\caption| 是放在 \verb|tabular| 或 \verb|tblr| 环境的上方。

\begin{figure}[htbp]
  \centering
  \includegraphics[width = 5cm]{example-image-a}
  \caption{测试}
  \label{figure:test}
\end{figure}

\begin{table}[htbp]
  \centering
  \caption{测试}
  \label{table:test}
  \begin{tabular}{|c|c|}
    11 & 22 \\
    33 & 44 
  \end{tabular}
\end{table}

图 \ref{figure:test} 和表 \ref{table:test} 用来测试两个浮动体和交叉引用



\section{部分数学符号的输入}

本节主要是一些数学符号的输入介绍


\subsection{直体符号}

科技类论文中,建议一些数学符号使用直体(“up”前缀表示直体)
  \begin{itemize}
    \item 直立的 pi :\verb|\uppi| $\to \uppi$
    \item 直立的 e :\verb|\upe| $\to \upe$
    \item 直立的 i :\verb|\upi| $\to \upi$
  \end{itemize}



\subsection{physics 宏包的一些命令}

\verb|CCNUthesis.cls| 已经默认加载了 \verb|physics| 宏包,提供了很多很方便的命令,本小节介绍部分,更多的请阅读宏包文档(命令行输入 \verb|texdoc physics|)

\begin{itemize}
  \item 向量 $\vb{a}, \vb*{a}, \va{a}, \va*{a}, \vu{a}, \vu*{a}$
  \item 积分
    \[
      \int_{0}^{1} x \dd{x},
      \dd[3]{x}, \dd(\cos\theta),
    \]
  \item 微分
    \[
      \dv{x}, \dv{f}{x}, \dv[n]{f}{x}, \dv*{f}{x}
    \]
  \item 偏微分
    \[
      \pdv{x}, \pdv{f}{x}, \pdv[n]{f}{x}
    \]
  \item 绝对值 $\abs{x^3}, \abs{\frac{1}{2}}$
  \item 范数 $\norm{f}, \norm{f_n}_{2}^{2}$
\end{itemize}   % 常用命令环境示例,不需要时注释掉即可
\chapter{引言}

\section{研究背景}
\begin{figure}[tbh]
  \centering
  \includegraphics[width = 5cm]{example-image-a}
  \caption{测试}
\end{figure}
\begin{table}[tbh]
  \caption{测试}
  \centering
  \begin{tblr}{|c|c|}
    11 & 22 \\
    33 & 44 
  \end{tblr}
\end{table}



\subsection{前人工作}


测试 \parencite{邱泽奇建构与分化}
\section{研究背景}

\chapter{预备知识}

\parencite[thm 3.1]{zurek2014quantum}

\parencite{zurek2014quantum}

\cite[test]{zurek2014quantum}

\cite{zurek2014quantum}


\section{基础公设}

整个量子力学的数学理论可以建立于五个基础公设。这些公设不能被严格推导出来的,而是从实验结果仔细分析
归纳总结而得到的。从这五个公设,可以推导出整个量子力学。假若量子力学的理论结果不符合实验结果,
则必须将这些基础公设加以修改,直到没有任何不符合之处。至今为止,量子力学已被实验核对至极高准确度,
还没有找到任何与理论不符合的实验结果,虽然有些理论很难直觉地用经典物理的概念来理解,例如,波粒
二象性、量子纠缠等等 \parencite{zurek2014quantum,cohen2013claude,zettili2003quantum}。

\begin{enumerate}
  \item 量子态公设:量子系统在任意时刻的状态(量子态)可以由希尔伯特空间 $\hilbertH$ 中的态矢量
    $\ket{\psi}$ 来设定,这态矢量完备地给出了这量子系统的所有信息。这公设意味着量子系统遵守%
    \emph{态叠加原理},假若 $\ket*{\psi_1}$、$\ket*{\psi_2}$ 属于希尔伯特空间 $\hilbertH$,则
    $c_1\ket*{\psi_1} + c_2\ket*{\psi_2}$ 也属于希尔伯特空间 $\hilbertH$。
  \item 时间演化公设: 态矢量为 $\ket{\psi(t)}$ 的量子系统,其动力学演化可以用薛定谔方程表示:
    \begin{equation}
      \upi\hbar \pdv{t} \ket{\psi(t)} = \hat{H} \ket{\psi(t)}.
    \end{equation}
    其中,哈密顿算符 $\hat{H}$ 对应于量子系统的总能量,$\hbar$ 是约化普朗克常数。根据薛定谔方程,
    假设时间从 $t_0$ 变化到 $t$,则态矢量从 $\ket*{\psi(t_0)}$ 演化到 $\ket{\psi(t)}$,该过程以
    方程表示为
    \begin{equation}
      \ket{\psi(t)} = \hat{U}(t,\,t_0) \ket*{\psi(t_0)}.
    \end{equation}
    其中 $\hat{U}(t,\,t_0) = \upe^{-\upi\hat{H}(t-t_0) / \hbar}$ 是时间演化算符。
  \item 可观察量公设:每个可观察量 $A$ 都有其对应的厄米算符 $\hat{A}$,而算符 $\hat{A}$ 的所有
    本征矢量共同组成一个完备基底。
  \item 坍缩公设:对于量子系统测量某个可观察量 $A$ 的过程,可以数学表示为将对应的厄米算符
    $\hat{A}$ 作用于量子系统的态矢量 $\ket{\psi}$,测量值只能为厄米算符 $\hat{A}$ 的本征值。
    在测量后,假设测量值为 $a_i$,则量子系统的量子态立刻会坍缩为对应于本征值 $a_i$ 的本征态
    $\ket*{e_i}$。
  \item 波恩公设:对于这测量,获得本征值 $a_i$ 的概率为量子态 $\ket{\psi}$ 处于本征态 $\ket*{e_i}$
    的概率幅的绝对值平方。\footnote{%
      使用可观察量 $A$ 的基底 $\qty{e_1,\,e_2,\,\ldots,\,e_n}$,量子态 $\ket{\psi}$ 可以表示为
      $\ket{\psi} = \sum_j c_j \ket*{e_j}$,其中 $c_j$ 是量子态 $\ket{\psi}$ 处于本征态
      $\ket*{e_j}$ 的概率幅。根据波恩定则,对于此次测量,获得本征值 $a_i$ 的概率为
      $\abs*{\ip*{e_i}{\psi}}^2 = \abs*{c_i}^2$。}
\end{enumerate}

\chapter{问题研究}
% !TeX root = ../main.tex

\chapter{总结与展望}



\section{}

\begin{equation}
  K \leq 
  \left\{ 
    \begin{array}{cl}
      2 \lfloor \frac{d}{2d - n} \rfloor    , & K \text{为偶数} ; \\
      2 \lfloor \frac{d}{2d - n} \rfloor - 1, & K \text{为奇数}. \\
    \end{array}
  \right. 
\end{equation}



% \backmatter开启后置部分,包含参考文献、致谢、附录等
\backmatter


%%%% 参考文献 %%%%
%%%%%%%%%%%%%%%%%%%%%%%%%%
%%% 行内引用:\parencite{} or \parencite[]{},下面两个情况要用行内引用
%    - 去掉这个引用句子结构不完整,比如“定理证明可参看[1]”
%    - 英文文献的引用
%%% 上标引用:\cite{} or \cite[]{},下面情况要用上标引用
%    - 去掉这个引用,句子结构完整,比如“作者提到,‘CCNUthesis真是一个好模版’^[1]”
%      其中“^[1]” 表示上标引用
%%%%%%%%%%%%%%%%%%%%%%%%%%
% 打印参考文献列表
\printbibliography



%%%% 致谢 %%%%
\Acknowledgements


感谢各位的使用,欢迎提出issue和bug!

\begin{signature}
  夏康玮 \\
  2022年4月21日于珞珈山
\end{signature}



%%%% 附录 %%%%
% 没有附录内容的把下面的代码注释掉即可
% !TeX root = ../main.tex

\appendix


\chapter{调查问卷}

用 \verb|choices| 环境可以排版 \emph{任意个} 选项,只需要像罗列环境 \verb|enumerate| 环境等一样用 \verb|\item| 分隔即可。

\verb|choices| 环境的 label 可以方便地进行调整
\begin{itemize}
  \item arabic(阿拉伯数字)
  \item alph(小写英文)
  \item Alph(大写英文)
  \item roman(小写罗马数字)
  \item Roman(大写罗马数字)
  \item circlednumber(带圈数字)
\end{itemize}

更多关于 \verb|choices| 环境的精细调整可以查看 \url{https://gitee.com/zepinglee/exam-zh}。

\begin{choices}[label = \arabic*)]
  \item 选项1
  \item 选项2
  \item 选项3
  \item 选项4
\end{choices}

\begin{choices}[label = (\alph*]
  \item 选项1
  \item 选项2
  \item 选项3
  \item 选项4
\end{choices}

\begin{choices}[label = \Alph*.]
  \item 选项1
  \item 选项2
  \item 选项3
  \item 选项4
\end{choices}

\begin{choices}[label = \roman*:]
  \item 选项1
  \item 选项2
  \item 选项3
  \item 选项4
\end{choices}

\begin{choices}[label = \Roman*-]
  \item 选项1
  \item 选项2
  \item 选项3
  \item 选项4
\end{choices}

\begin{choices}[label = \circlednumber*]
  \item 选项1
  \item 选项2
  \item 选项3
  \item 选项4
  \item 选项5
  \item 选项6
  \item 选项7
  \item 选项8
\end{choices}


还可以修改 \verb|columns| 键值来决定每行排多少个
\begin{choices}[
  columns = 3,            % 手动控制每行多少个选项,否则自己根据宽度自动排版
  label = (\arabic*)      % label 的样式,支持 arabic, alph, Alph, roman, Roman, circlednumber
]
  \item 选项1
  \item 选项2
  \item 选项3
  \item 选项4
  \item 选项5
  \item 选项6
\end{choices}



\chapter{访谈记录}


\section{与 A 的访谈记录}

\begin{figure}[htbp]
  \centering
  \includegraphics[width = 5cm]{example-image-a}
  \caption{测试}
  \label{figure:test2}
\end{figure}

\begin{table}[htbp]
  \centering
  \caption{测试}
  \label{table:test2}
  \begin{tabular}{|c|c|}
    11 & 22 \\
    33 & 44 
  \end{tabular}
\end{table}

\begin{figure}[htbp]
  \centering
  \includegraphics[width = 5cm]{example-image-a}
  \caption{测试}
  \label{figure:test3}
\end{figure}

\begin{table}[htbp]
  \centering
  \caption{测试}
  \label{table:test3}
  \begin{tabular}{|c|c|}
    11 & 22 \\
    33 & 44 
  \end{tabular}
\end{table}

\section{与 B 的访谈记录}


\chapter{访谈记录}


\section{与 A 的访谈记录}

\begin{figure}[htbp]
  \centering
  \includegraphics[width = 5cm]{example-image-a}
  \caption{测试}
  \label{figure:test4}
\end{figure}

\begin{table}[htbp]
  \centering
  \caption{测试}
  \label{table:test4}
  \begin{tabular}{|c|c|}
    11 & 22 \\
    33 & 44 
  \end{tabular}
\end{table}

\begin{figure}[htbp]
  \centering
  \includegraphics[width = 5cm]{example-image-a}
  \caption{测试}
  \label{figure:test5}
\end{figure}

\begin{table}[htbp]
  \centering
  \caption{测试}
  \label{table:test5}
  \begin{tabular}{|c|c|}
    11 & 22 \\
    33 & 44 
  \end{tabular}
\end{table}

\section{与 B 的访谈记录}

\end{document}
