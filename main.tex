% !TeX encoding = UTF-8
% !TeX program = xelatex
% !TeX spellcheck = en_US

%********************************************
% CCNUthesis: 华中师范大学论文模板
% 2022/05/22 v1.2.1
%
% 重要提示:
%   1. 请确保使用 UTF-8 编码保存
%   2. 请使用 XeLaTeX 或 latexmk 编译
%   3. 请仔细阅读用户文档
%   4. 不需要的注释可以尽情删除
%********************************************



% 交终稿的时候需要单面打印,设置 oneside,奇偶页页面设置相同
\documentclass[oneside]{CCNUthesis}

% 答辩打印可以双面打印,设置 twoside,奇偶页页面设置不同
% \documentclass[twoside]{CCNUthesis}


\ccnusetup{
  % 个人信息
  info = {
    % 中文标题,会自动换行
    % 如果换行点不满意,可以用 \\ 手动换行
    title = {
      如何使用 \LaTeX{} 制作一个学位论文模版 \\ 
      ——以华中师范大学为例
    },
    % 论文的英文标题
    title* = {
      Thesis Title
    }, 
    % 学院
    department = {数学与统计学学院},
    % department = {数学},
    % 专业
    major = {数学与应用数学(试验)},
    % major = {数学},
    % 年级
    level = {2018级},
    % 姓名
    author = {你的姓名},
    % 学号
    student-id = {学号},
    % 指导教师
    supervisor = {教师姓名 \quad 职称},
    % 论文中文关键词,用英文“,”隔开
    keywords = {
      关键词1,
      关键词2,
      关键词3
    },
    % 论文英文关键词,用英文“,”隔开
    keywords* = {
      keyword1,
      keyword2,
      keyword3
    },
    % 如有需要可以手动调整封面底的年和月,否则默认为「编译时」的年和月
    % year  = {2022},
    % month = {4}
  },
  style = {
    font = times,
      % 西文字体
      % 允许选项
      % font = newtx|times|stixtwo|xits|tg|none
    cjk-font = fandol,
      % 中文字体
      % 允许选项:
      %   cjk-font = adobe|fandol|founder|mac|sinotype|sourcehan|windows|none
      % 注意:
      %   1. 中文字体设置高度依赖于系统。各系统建议方案:
      %        windows:cjk-font = windows
      %        mac:    cjk-font = mac
      %        linux:  cjk-font = fandol(默认值)
      %   2. 除 fandol 和 sourcehan 外,其余字体均为商用字体,请注意版权问题
      %   3. 但 fandol 字体缺字比较严重,而 sourcehan 没有配备楷体和仿宋体
    caption-labelstyle = hyphen,
      % 图表的标题 label 计数样式
      % 允许选项:
      %    arabic|hyphen
      %      arabic:样式为图1,图2,表1,表2...,并且跨 chapter 连续编号,即 上一个 chapter 的图编号若为4,下一个 chapter 的第一个图编号为 5
      %      hyphen:样式为图1-1,图2-1,表1-1...
      %      x-y 的 x 为 chapter 值,y 为图表的计数器值,新的 chapter 中 y 会清零从新计数
    caption-labelseperator = colon,
      % 图表标题 label 和 标题内容 之内的分隔符
      % 允许选项:
      %    colon|space
      %      colon 表示 「:␣」,即一个西文冒号加一个空格
      %      space 表示 「␣␣」,即两个空格
    bib-style = ccnu-author-year,
      % bib-style 表示参考文献的格式
      % 允许的选项:
      %   ccnu-numerical|ccnu-author-year|gb7714-2015
      %   ccnu-numerical:学校标准、按照引用顺序排
      %   ccnu-author-year:学校标准、按照作者名、年份排
      %   gb7714-2015:国标
    bib-resource = {CCNUthesis-main.bib},
      % 参考文献数据源,需要加bib后缀
    fullwidth-stop = catcode,
      % 是否把全角实心句点 “.” 作为默认的句号形状
      % 即正文中输入“。” 最终编译效果为“. ”
      % 一般科技类文章需要替换,防止“. ”与“。”混淆
      % 允许选项:
      %   fullwidth-stop = catcode|mapping|false
      % 说明:
      %   catcode   显式的 “。” 会被替换为 “.”(e.g. 不包括用宏定义保存的 “。”)
      %   mapping   所有的 “。” 会被替换为 “.”(使用 LuaLaTeX 编译则无效)
      %   false     不进行替换
  }
}



%%%%% 需要的宏包可以在此处自行调用 %%%%%
\usepackage{mathtools}


% 需要的命令环境可以自行定义
\newcommand{\upe}{\mathrm{e}}        % 直立的e,用于表示常量,如自然常数      
\newcommand{\upi}{\mathrm{i}}        % 直立的i,用于表示常量,如虚数单位
\newcommand{\hilbertH}{\mathcal{H}}



\begin{document}


% \frontmatter 开启论文前置部分
% 前置部分包含目录、中英文摘要以及符号表等
\frontmatter

% 目录
\tableofcontents


% 摘要
% !TeX root = ../main.tex

% 中文摘要
\begin{abstract}
  论文的摘要是对论文研究内容和成果的高度概括。
  摘要应对论文所研究的问题及其研究目的进行描述,对研究方法和过程进行简单介绍,对研究成果和所得结论进行概括。
  摘要应具有独立性和自明性,其内容应包含与论文全文同等量的主要信息。
  使读者即使不阅读全文,通过摘要就能了解论文的总体内容和主要成果。

  论文摘要的书写应力求精确、简明。
  切忌写成对论文书写内容进行提要的形式,尤其要避免“第 1 章……;第 2 章……;……”这种或类似的陈述方式。

  关键词是为了文献标引工作、用以表示全文主要内容信息的单词或术语。

  \zhlipsum[1-5]
  \zhlipsum[1-5]
  \zhlipsum[1-5]
\end{abstract}



% 英文摘要
\begin{abstract*}
  An abstract of a dissertation is a summary and extraction of research work and contributions. Included in an abstract should be description of research topic and research objective, brief introduction to methodology and research process, and summary of conclusion and contributions of the research. An abstract should be characterized by independence and clarity and carry identical information with the dissertation. It should be such that the general idea and major contributions of the dissertation are conveyed without reading the dissertation.

  An abstract should be concise and to the point. It is a misunderstanding to make an abstract an outline of the dissertation and words ``the first chapter'', ``the second chapter'' and the like should be avoided in the abstract.

  Keywords are terms used in a dissertation for indexing, reflecting core information of the dissertation. An abstract may contain a maximum of 5 keywords, with semi-colons used in between to separate one another.

  \zhlipsum[1-5]
  \zhlipsum[1-5]
  \zhlipsum[1-5]

\end{abstract*}


% 符号表,不需要的话将下面的代码注释掉即可
% !TeX root = ../main.tex

% 符号表

% 语法与 LaTeX 表格一致:列用 & 区分,行用 \\ 区分
% 如需修改格式,可以使用可选参数:
%   \begin{notation}[ll]
%     $x$ & 坐标 \\
%     $p$ & 动量
%   \end{notation}
% 可选参数与 LaTeX 标准表格的列格式说明语法一致
% 这里的 “ll” 表示两列均为自动宽度,并且左对齐


\begin{notation}
  $x$                  & 坐标        \\
  $p$                  & 动量        \\
  $\psi(x)$            & 波函数      \\
\end{notation}



% \mainmatter 进入论文主体部分
% 主体部分是论文的核心
\mainmatter

% 主体采用多文件编译的方式
% 即把每一章放进一个单独的 tex 文件里,并在这里用 \input 导入
% 例如 \chapter{引言}

\section{研究背景}
\begin{figure}[tbh]
  \centering
  \includegraphics[width = 5cm]{example-image-a}
  \caption{测试}
\end{figure}
\begin{table}[tbh]
  \caption{测试}
  \centering
  \begin{tblr}{|c|c|}
    11 & 22 \\
    33 & 44 
  \end{tblr}
\end{table}



\subsection{前人工作}


测试 \parencite{邱泽奇建构与分化}
\section{研究背景}

% 表示插入 main 所在目录中的 body 目录下的 chapter1.tex 文件


\chapter{引言}

\section{研究背景}
\begin{figure}[tbh]
  \centering
  \includegraphics[width = 5cm]{example-image-a}
  \caption{测试}
\end{figure}
\begin{table}[tbh]
  \caption{测试}
  \centering
  \begin{tblr}{|c|c|}
    11 & 22 \\
    33 & 44 
  \end{tblr}
\end{table}



\subsection{前人工作}


测试 \parencite{邱泽奇建构与分化}
\section{研究背景}

\chapter{预备知识}

\parencite[thm 3.1]{zurek2014quantum}

\parencite{zurek2014quantum}

\cite[test]{zurek2014quantum}

\cite{zurek2014quantum}


\section{基础公设}

整个量子力学的数学理论可以建立于五个基础公设。这些公设不能被严格推导出来的,而是从实验结果仔细分析
归纳总结而得到的。从这五个公设,可以推导出整个量子力学。假若量子力学的理论结果不符合实验结果,
则必须将这些基础公设加以修改,直到没有任何不符合之处。至今为止,量子力学已被实验核对至极高准确度,
还没有找到任何与理论不符合的实验结果,虽然有些理论很难直觉地用经典物理的概念来理解,例如,波粒
二象性、量子纠缠等等 \parencite{zurek2014quantum,cohen2013claude,zettili2003quantum}。

\begin{enumerate}
  \item 量子态公设:量子系统在任意时刻的状态(量子态)可以由希尔伯特空间 $\hilbertH$ 中的态矢量
    $\ket{\psi}$ 来设定,这态矢量完备地给出了这量子系统的所有信息。这公设意味着量子系统遵守%
    \emph{态叠加原理},假若 $\ket*{\psi_1}$、$\ket*{\psi_2}$ 属于希尔伯特空间 $\hilbertH$,则
    $c_1\ket*{\psi_1} + c_2\ket*{\psi_2}$ 也属于希尔伯特空间 $\hilbertH$。
  \item 时间演化公设: 态矢量为 $\ket{\psi(t)}$ 的量子系统,其动力学演化可以用薛定谔方程表示:
    \begin{equation}
      \upi\hbar \pdv{t} \ket{\psi(t)} = \hat{H} \ket{\psi(t)}.
    \end{equation}
    其中,哈密顿算符 $\hat{H}$ 对应于量子系统的总能量,$\hbar$ 是约化普朗克常数。根据薛定谔方程,
    假设时间从 $t_0$ 变化到 $t$,则态矢量从 $\ket*{\psi(t_0)}$ 演化到 $\ket{\psi(t)}$,该过程以
    方程表示为
    \begin{equation}
      \ket{\psi(t)} = \hat{U}(t,\,t_0) \ket*{\psi(t_0)}.
    \end{equation}
    其中 $\hat{U}(t,\,t_0) = \upe^{-\upi\hat{H}(t-t_0) / \hbar}$ 是时间演化算符。
  \item 可观察量公设:每个可观察量 $A$ 都有其对应的厄米算符 $\hat{A}$,而算符 $\hat{A}$ 的所有
    本征矢量共同组成一个完备基底。
  \item 坍缩公设:对于量子系统测量某个可观察量 $A$ 的过程,可以数学表示为将对应的厄米算符
    $\hat{A}$ 作用于量子系统的态矢量 $\ket{\psi}$,测量值只能为厄米算符 $\hat{A}$ 的本征值。
    在测量后,假设测量值为 $a_i$,则量子系统的量子态立刻会坍缩为对应于本征值 $a_i$ 的本征态
    $\ket*{e_i}$。
  \item 波恩公设:对于这测量,获得本征值 $a_i$ 的概率为量子态 $\ket{\psi}$ 处于本征态 $\ket*{e_i}$
    的概率幅的绝对值平方。\footnote{%
      使用可观察量 $A$ 的基底 $\qty{e_1,\,e_2,\,\ldots,\,e_n}$,量子态 $\ket{\psi}$ 可以表示为
      $\ket{\psi} = \sum_j c_j \ket*{e_j}$,其中 $c_j$ 是量子态 $\ket{\psi}$ 处于本征态
      $\ket*{e_j}$ 的概率幅。根据波恩定则,对于此次测量,获得本征值 $a_i$ 的概率为
      $\abs*{\ip*{e_i}{\psi}}^2 = \abs*{c_i}^2$。}
\end{enumerate}

\chapter{问题研究}
% !TeX root = ../main.tex

\chapter{总结与展望}



\section{}

\begin{equation}
  K \leq 
  \left\{ 
    \begin{array}{cl}
      2 \lfloor \frac{d}{2d - n} \rfloor    , & K \text{为偶数} ; \\
      2 \lfloor \frac{d}{2d - n} \rfloor - 1, & K \text{为奇数}. \\
    \end{array}
  \right. 
\end{equation}



% 论文后文部分,参考文献、致谢、附录等
\backmatter

%%%% 参考文献 %%%%
%%%%%%%%%%%%%%%%%%%%%%%%%%
%%% 行内引用:\parencite{} or \parencite[]{},下面两个情况要用行内引用
%    - 去掉这个引用句子结构不完整,比如“定理证明可参看[1]”
%    - 英文文献的引用
%%% 上标引用:\cite{} or \cite[]{},下面情况要用上标引用
%    - 去掉这个引用,句子结构完整,比如“作者提到,‘CCNUthesis 真是一个好模版。’^[1]”
%      其中“^[1]” 表示上标引用
%%%%%%%%%%%%%%%%%%%%%%%%%%
% 打印参考文献列表
\printbibliography


%%%% 致谢 %%%%
\Acknowledgements


感谢各位的使用,欢迎提出issue和bug!

\begin{signature}
  夏康玮 \\
  2022年4月21日于珞珈山
\end{signature}


%%%% 附录 %%%%
% !TeX root = ../main.tex

\appendix


\chapter{调查问卷}

用 \verb|choices| 环境可以排版 \emph{任意个} 选项,只需要像罗列环境 \verb|enumerate| 环境等一样用 \verb|\item| 分隔即可。

\verb|choices| 环境的 label 可以方便地进行调整
\begin{itemize}
  \item arabic(阿拉伯数字)
  \item alph(小写英文)
  \item Alph(大写英文)
  \item roman(小写罗马数字)
  \item Roman(大写罗马数字)
  \item circlednumber(带圈数字)
\end{itemize}

更多关于 \verb|choices| 环境的精细调整可以查看 \url{https://gitee.com/zepinglee/exam-zh}。

\begin{choices}[label = \arabic*)]
  \item 选项1
  \item 选项2
  \item 选项3
  \item 选项4
\end{choices}

\begin{choices}[label = (\alph*]
  \item 选项1
  \item 选项2
  \item 选项3
  \item 选项4
\end{choices}

\begin{choices}[label = \Alph*.]
  \item 选项1
  \item 选项2
  \item 选项3
  \item 选项4
\end{choices}

\begin{choices}[label = \roman*:]
  \item 选项1
  \item 选项2
  \item 选项3
  \item 选项4
\end{choices}

\begin{choices}[label = \Roman*-]
  \item 选项1
  \item 选项2
  \item 选项3
  \item 选项4
\end{choices}

\begin{choices}[label = \circlednumber*]
  \item 选项1
  \item 选项2
  \item 选项3
  \item 选项4
  \item 选项5
  \item 选项6
  \item 选项7
  \item 选项8
\end{choices}


还可以修改 \verb|columns| 键值来决定每行排多少个
\begin{choices}[
  columns = 3,            % 手动控制每行多少个选项,否则自己根据宽度自动排版
  label = (\arabic*)      % label 的样式,支持 arabic, alph, Alph, roman, Roman, circlednumber
]
  \item 选项1
  \item 选项2
  \item 选项3
  \item 选项4
  \item 选项5
  \item 选项6
\end{choices}



\chapter{访谈记录}


\section{与 A 的访谈记录}

\begin{figure}[htbp]
  \centering
  \includegraphics[width = 5cm]{example-image-a}
  \caption{测试}
  \label{figure:test2}
\end{figure}

\begin{table}[htbp]
  \centering
  \caption{测试}
  \label{table:test2}
  \begin{tabular}{|c|c|}
    11 & 22 \\
    33 & 44 
  \end{tabular}
\end{table}

\begin{figure}[htbp]
  \centering
  \includegraphics[width = 5cm]{example-image-a}
  \caption{测试}
  \label{figure:test3}
\end{figure}

\begin{table}[htbp]
  \centering
  \caption{测试}
  \label{table:test3}
  \begin{tabular}{|c|c|}
    11 & 22 \\
    33 & 44 
  \end{tabular}
\end{table}

\section{与 B 的访谈记录}


\chapter{访谈记录}


\section{与 A 的访谈记录}

\begin{figure}[htbp]
  \centering
  \includegraphics[width = 5cm]{example-image-a}
  \caption{测试}
  \label{figure:test4}
\end{figure}

\begin{table}[htbp]
  \centering
  \caption{测试}
  \label{table:test4}
  \begin{tabular}{|c|c|}
    11 & 22 \\
    33 & 44 
  \end{tabular}
\end{table}

\begin{figure}[htbp]
  \centering
  \includegraphics[width = 5cm]{example-image-a}
  \caption{测试}
  \label{figure:test5}
\end{figure}

\begin{table}[htbp]
  \centering
  \caption{测试}
  \label{table:test5}
  \begin{tabular}{|c|c|}
    11 & 22 \\
    33 & 44 
  \end{tabular}
\end{table}

\section{与 B 的访谈记录}



\end{document}