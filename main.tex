\documentclass{CCNUthesis}

\ccnusetup{
  % style 类用于设置论文格式
  style = {
    cjk-font = fandol,
    % 中文字体
    % 允许选项:
    %   cjk-font = adobe|fandol|founder|mac|sinotype|sourcehan|windows|none
    % 注意:
    %   1. 中文字体设置高度依赖于系统。各系统建议方案:
    %        windows:cjk-font = windows
    %        mac:    cjk-font = mac
    %        linux:  cjk-font = fandol(默认值)
    %   2. 除 fandol 和 sourcehan 外,其余字体均为商用字体,请注意版权问题
    %   3. 但 fandol 字体缺字比较严重,而 sourcehan 没有配备楷体和仿宋体
    },
  % 个人信息
  info = {
    % 论文的中文标题
    title = {Lipschitz函数的若干性质},
    % 论文的英文标题
    title* = {Some properties of the Lipschitz functions
    }, 
    % 学院
    major = {数学与统计学学院},
    % 专业
    department = {数学与应用数学(试验)},
    % 年级
    level = {2017},
    % 姓名
    author = {夏康玮},
    % 学号
    student-id = {2017214294},
    % 指导教师
    supervisor = {李工宝 \quad 教授},
    % 论文中文关键词,用英文“,”隔开
    keywords = {Lipschitz函数, 可微性, Hausdorff测度, Hausdorff维数},
    % 论文英文关键词,用英文“,”隔开
    keywords* = {Lipschitz functions, Differentiability, Hausdorff measure, Hausdorff dimension} 
  }
}

% 需要的宏包可以自行调用
\usepackage{physics}   % 仅为示例用,可删除

% 需要的命令可以自行定义
\newcommand{\hilbertH}{\symcal{H}}   % 仅为示例用,可删除
\newcommand{\ee}{\symrm{e}}   % 仅为示例用,可删除
\newcommand{\ii}{\symrm{i}}   % 仅为示例用,可删除

\begin{document}
% 前置部分包含目录、中英文摘要以及符号表等
\frontmatter

\ccnusetup{
  style = { fullwidth-stop = catcode }
    % 是否把全角实心句点 “.” 作为默认的句号形状,即正文中输入“。” 最终编译效果为“. ”
    % 一般科技类文章需要替换,防止“. ”与“。”混淆
    % 放在此处是为了防止干扰版权页的句号设置
    % 允许选项:
    %   fullwidth-stop = catcode|mapping|false
    % 说明:
    %   catcode   显式的 “。” 会被替换为 “.”(e.g. 不包括用宏定义保存的 “。”)
    %   mapping   所有的 “。” 会被替换为 “.”(使用 LuaLaTeX 编译则无效)
    %   false     不进行替换
}

% 目录
\tableofcontents


% 中文摘要
\begin{abstract}
  Lipschitz函数作为1阶H\"older连续函数的像集与图像的Hausdorff测度和Hausdorff维数具有特殊的性质; Rademacher证明了 $R^n$ 上的Lipschitz函数的可微性.  本文主要通过系统梳理测度论有关基础理论知识并对参考书籍中一些必要的或是欠严谨的地方进行适当补充, 给出Rademacher定理以及Hausdorff测度有关性质相对自封闭的证明, 使得学过实变函数论的本科生能够较为容易地理解与学习。
\end{abstract}

% 英文摘要
\begin{abstract*}
  As H\"older continuous functions with exponent 1, the Lipschitz functions' images and graphes have some special properties with respect to their Hausdorff measures and Hausdorff dimensions; the differentiability of Lipschitz functions on $R^n$ have been proved by Rademacher.  By systematically sorting out the basic theoretical knowledge of measure theory and appropriately supplementing some necessary or less rigorous points in the books referred to, we provide relatively self-closed proofs of Rademacher's Theorem and some properties involving Hausdorff measures, which makes it easier for undergraduates who have learned the theory of real functions to understand and learn.
\end{abstract*}

% 符号表
% 语法与 LaTeX 表格一致:列用 & 区分,行用 \\ 区分
% 如需修改格式,可以使用可选参数:
%   \begin{notation}[ll]
%     $x$ & 坐标 \\
%     $p$ & 动量
%   \end{notation}
% 可选参数与 LaTeX 标准表格的列格式说明语法一致
% 这里的 “ll” 表示两列均为自动宽度,并且左对齐
\begin{notation}[ll]
  $x$                  & 坐标        \\
  $p$                  & 动量        \\
  $\psi(x)$            & 波函数      \\
\end{notation}



% 主体部分是论文的核心
\mainmatter


% 采用多文件编译的方式
% 把每一章放进一个单独的 tex 文件里,并在这里用 \include 导入
% 例如 \chapter{引言}

\section{研究背景}
\begin{figure}[tbh]
  \centering
  \includegraphics[width = 5cm]{example-image-a}
  \caption{测试}
\end{figure}
\begin{table}[tbh]
  \caption{测试}
  \centering
  \begin{tblr}{|c|c|}
    11 & 22 \\
    33 & 44 
  \end{tblr}
\end{table}



\subsection{前人工作}


测试 \parencite{邱泽奇建构与分化}
\section{研究背景}

% 表示插入main所在目录中的body目录下的chapter1.tex文件

% 章节一般包括【引言】、【预备知识】以及最后的【总结与展望】
% 中间部分灵活掌握
% 文件中的标题仅为示例,可根据自己需要进行修改
\chapter{引言}

\section{研究背景}
\begin{figure}[tbh]
  \centering
  \includegraphics[width = 5cm]{example-image-a}
  \caption{测试}
\end{figure}
\begin{table}[tbh]
  \caption{测试}
  \centering
  \begin{tblr}{|c|c|}
    11 & 22 \\
    33 & 44 
  \end{tblr}
\end{table}



\subsection{前人工作}


测试 \parencite{邱泽奇建构与分化}
\section{研究背景}

\chapter{预备知识}

\parencite[thm 3.1]{zurek2014quantum}

\parencite{zurek2014quantum}

\cite[test]{zurek2014quantum}

\cite{zurek2014quantum}


\section{基础公设}

整个量子力学的数学理论可以建立于五个基础公设。这些公设不能被严格推导出来的,而是从实验结果仔细分析
归纳总结而得到的。从这五个公设,可以推导出整个量子力学。假若量子力学的理论结果不符合实验结果,
则必须将这些基础公设加以修改,直到没有任何不符合之处。至今为止,量子力学已被实验核对至极高准确度,
还没有找到任何与理论不符合的实验结果,虽然有些理论很难直觉地用经典物理的概念来理解,例如,波粒
二象性、量子纠缠等等 \parencite{zurek2014quantum,cohen2013claude,zettili2003quantum}。

\begin{enumerate}
  \item 量子态公设:量子系统在任意时刻的状态(量子态)可以由希尔伯特空间 $\hilbertH$ 中的态矢量
    $\ket{\psi}$ 来设定,这态矢量完备地给出了这量子系统的所有信息。这公设意味着量子系统遵守%
    \emph{态叠加原理},假若 $\ket*{\psi_1}$、$\ket*{\psi_2}$ 属于希尔伯特空间 $\hilbertH$,则
    $c_1\ket*{\psi_1} + c_2\ket*{\psi_2}$ 也属于希尔伯特空间 $\hilbertH$。
  \item 时间演化公设: 态矢量为 $\ket{\psi(t)}$ 的量子系统,其动力学演化可以用薛定谔方程表示:
    \begin{equation}
      \upi\hbar \pdv{t} \ket{\psi(t)} = \hat{H} \ket{\psi(t)}.
    \end{equation}
    其中,哈密顿算符 $\hat{H}$ 对应于量子系统的总能量,$\hbar$ 是约化普朗克常数。根据薛定谔方程,
    假设时间从 $t_0$ 变化到 $t$,则态矢量从 $\ket*{\psi(t_0)}$ 演化到 $\ket{\psi(t)}$,该过程以
    方程表示为
    \begin{equation}
      \ket{\psi(t)} = \hat{U}(t,\,t_0) \ket*{\psi(t_0)}.
    \end{equation}
    其中 $\hat{U}(t,\,t_0) = \upe^{-\upi\hat{H}(t-t_0) / \hbar}$ 是时间演化算符。
  \item 可观察量公设:每个可观察量 $A$ 都有其对应的厄米算符 $\hat{A}$,而算符 $\hat{A}$ 的所有
    本征矢量共同组成一个完备基底。
  \item 坍缩公设:对于量子系统测量某个可观察量 $A$ 的过程,可以数学表示为将对应的厄米算符
    $\hat{A}$ 作用于量子系统的态矢量 $\ket{\psi}$,测量值只能为厄米算符 $\hat{A}$ 的本征值。
    在测量后,假设测量值为 $a_i$,则量子系统的量子态立刻会坍缩为对应于本征值 $a_i$ 的本征态
    $\ket*{e_i}$。
  \item 波恩公设:对于这测量,获得本征值 $a_i$ 的概率为量子态 $\ket{\psi}$ 处于本征态 $\ket*{e_i}$
    的概率幅的绝对值平方。\footnote{%
      使用可观察量 $A$ 的基底 $\qty{e_1,\,e_2,\,\ldots,\,e_n}$,量子态 $\ket{\psi}$ 可以表示为
      $\ket{\psi} = \sum_j c_j \ket*{e_j}$,其中 $c_j$ 是量子态 $\ket{\psi}$ 处于本征态
      $\ket*{e_j}$ 的概率幅。根据波恩定则,对于此次测量,获得本征值 $a_i$ 的概率为
      $\abs*{\ip*{e_i}{\psi}}^2 = \abs*{c_i}^2$。}
\end{enumerate}

\chapter{问题研究}
% !TeX root = ../main.tex

\chapter{总结与展望}



\section{}

\begin{equation}
  K \leq 
  \left\{ 
    \begin{array}{cl}
      2 \lfloor \frac{d}{2d - n} \rfloor    , & K \text{为偶数} ; \\
      2 \lfloor \frac{d}{2d - n} \rfloor - 1, & K \text{为奇数}. \\
    \end{array}
  \right. 
\end{equation}



% 后置部分包含参考文献、声明页(自动生成)等
\backmatter


% 参考文献只需要修改CCNUthesis-main.bib文件
% 打印参考文献列表
\printbibliography

% 致谢
\Acknowledgements


感谢各位的使用,欢迎提出issue和bug!

\begin{signature}
  夏康玮 \\
  2022年4月21日于珞珈山
\end{signature}
% 附录
% !TeX root = ../main.tex

\appendix


\chapter{调查问卷}

用 \verb|choices| 环境可以排版 \emph{任意个} 选项,只需要像罗列环境 \verb|enumerate| 环境等一样用 \verb|\item| 分隔即可。

\verb|choices| 环境的 label 可以方便地进行调整
\begin{itemize}
  \item arabic(阿拉伯数字)
  \item alph(小写英文)
  \item Alph(大写英文)
  \item roman(小写罗马数字)
  \item Roman(大写罗马数字)
  \item circlednumber(带圈数字)
\end{itemize}

更多关于 \verb|choices| 环境的精细调整可以查看 \url{https://gitee.com/zepinglee/exam-zh}。

\begin{choices}[label = \arabic*)]
  \item 选项1
  \item 选项2
  \item 选项3
  \item 选项4
\end{choices}

\begin{choices}[label = (\alph*]
  \item 选项1
  \item 选项2
  \item 选项3
  \item 选项4
\end{choices}

\begin{choices}[label = \Alph*.]
  \item 选项1
  \item 选项2
  \item 选项3
  \item 选项4
\end{choices}

\begin{choices}[label = \roman*:]
  \item 选项1
  \item 选项2
  \item 选项3
  \item 选项4
\end{choices}

\begin{choices}[label = \Roman*-]
  \item 选项1
  \item 选项2
  \item 选项3
  \item 选项4
\end{choices}

\begin{choices}[label = \circlednumber*]
  \item 选项1
  \item 选项2
  \item 选项3
  \item 选项4
  \item 选项5
  \item 选项6
  \item 选项7
  \item 选项8
\end{choices}


还可以修改 \verb|columns| 键值来决定每行排多少个
\begin{choices}[
  columns = 3,            % 手动控制每行多少个选项,否则自己根据宽度自动排版
  label = (\arabic*)      % label 的样式,支持 arabic, alph, Alph, roman, Roman, circlednumber
]
  \item 选项1
  \item 选项2
  \item 选项3
  \item 选项4
  \item 选项5
  \item 选项6
\end{choices}



\chapter{访谈记录}


\section{与 A 的访谈记录}

\begin{figure}[htbp]
  \centering
  \includegraphics[width = 5cm]{example-image-a}
  \caption{测试}
  \label{figure:test2}
\end{figure}

\begin{table}[htbp]
  \centering
  \caption{测试}
  \label{table:test2}
  \begin{tabular}{|c|c|}
    11 & 22 \\
    33 & 44 
  \end{tabular}
\end{table}

\begin{figure}[htbp]
  \centering
  \includegraphics[width = 5cm]{example-image-a}
  \caption{测试}
  \label{figure:test3}
\end{figure}

\begin{table}[htbp]
  \centering
  \caption{测试}
  \label{table:test3}
  \begin{tabular}{|c|c|}
    11 & 22 \\
    33 & 44 
  \end{tabular}
\end{table}

\section{与 B 的访谈记录}


\chapter{访谈记录}


\section{与 A 的访谈记录}

\begin{figure}[htbp]
  \centering
  \includegraphics[width = 5cm]{example-image-a}
  \caption{测试}
  \label{figure:test4}
\end{figure}

\begin{table}[htbp]
  \centering
  \caption{测试}
  \label{table:test4}
  \begin{tabular}{|c|c|}
    11 & 22 \\
    33 & 44 
  \end{tabular}
\end{table}

\begin{figure}[htbp]
  \centering
  \includegraphics[width = 5cm]{example-image-a}
  \caption{测试}
  \label{figure:test5}
\end{figure}

\begin{table}[htbp]
  \centering
  \caption{测试}
  \label{table:test5}
  \begin{tabular}{|c|c|}
    11 & 22 \\
    33 & 44 
  \end{tabular}
\end{table}

\section{与 B 的访谈记录}



\end{document}